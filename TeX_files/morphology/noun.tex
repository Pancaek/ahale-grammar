\chapter{Nominal}
\section{Alignment}

\langname\ nouns are quite analytic.
A typical noun consists of a stem, plus an affix denoting case, and optional plural marking.
This may be either ergative or absolutive, though the absolutive is unmarked.
The following points give an indication of when each case should be used.
\begin{itemize}
  \item The agent of a transitive verb (A) is marked with ergative case
  \item The core argument of an intransitive verb (S) and the patient of a transitive verb (P) are both marked with absolutive case.
\end{itemize}

\pex<alignment>
\a<itrns>
  \begingl
  \glpreamble hawi keke
  \pronounced{ˈɣa.wi ˈke.kə}\endpreamble
  Ø-hawi[\textsc{abs-}rabbit]
  keke[\textsc{npst.ipfv-}eat]
  \glft `The rabbit is eating'
  \endgl
  \a<trns>
  \begingl
  \glpreamble hahawi keke ɸumau
  \pronounced{ˈɣa.ɣa.wi ˈke.kə ˈɸu.mau}\endpreamble
  ha\textasciitilde hawi[\textsc{erg\textasciitilde}rabbit]
  keke[\textsc{npst.ipfv-}eat]
  ɸumau[\textsc{abs,}grass]
  \glft `The rabbit is eating the grass'
  \endgl
  \xe

Note that the stressed allophone of /h/ remains when reduplicated, this will be further explored in the following sections.

Notice that ``rabbit'' is declined in a different case for these two similar sentences.
The ergative is marked through reduplication of the first syllable.

\section{Plurality}

Plurals are formed with an affix \langword{me-}.
To illustrate its use we can revisit Example \getfullref{alignment.trns}.
`\langword{hahawi keke ɸumau}'.
If we want to pluralize \langword{hahawi,} we may expect \langword{mehahawi.} This is not the case however.
\langword{-me} is inserted between the root and the reduplicated ergative marking.
The reduplicated segment is not changed though, so the correct plural is \langword{hamehawi.} \langword{me-} is most accurately described as an interfix.

\pex<vowel-initial>
\a
\begingl
\glpreamble ana
\pronounced{ˈa.na}\endpreamble
Ø-ana[\textsc{abs-}eye]
\endgl

\a<epenthesis>
\begingl
\glpreamble aʔana
\pronounced{ˈa.ʔa.na}\endpreamble
aʔ\textasciitilde ana[\textsc{erg\textasciitilde }eye]
\endgl

\a
\begingl
\glpreamble meana
\pronounced{ˈme.a.na}\endpreamble
<me>Ø-ana[\textsc{<pl>abs-}eye]
\endgl

\a
\begingl
\glpreamble ameana
\pronounced{ˈa.mə.a.na}\endpreamble
a<me>ana[\textsc{erg<pl>\textasciitilde}eye]
\endgl
\xe
\pex<consonant-initial>
\a
\begingl
\glpreamble hawi
\pronounced{ˈɣa.wi}\endpreamble
Ø-hawi[\textsc{abs-}rabbit]
\endgl

\a
\begingl
\glpreamble hahawi
\pronounced{ˈɣa.ɣa.wi}\endpreamble
ha\textasciitilde hawi[\textsc{erg\textasciitilde }rabbit]
\endgl

\a
\begingl
\glpreamble mehawi
\pronounced{ˈme.ha.wi}\endpreamble
<me>Ø-hawi[\textsc{<pl>abs-}rabbit]
\endgl

\a
\begingl
\glpreamble hamehawi
\pronounced{ˈɣa.mə.ɣa.wi}\endpreamble
ha<me>hawi[\textsc{erg<pl>\textasciitilde}rabbit]
\endgl
\xe

Note how many forms of \langword{hawi} maintain [ɣ] even in unstressed envirenments.
This is because the reduplication causes it to be preserved.

In Example \getfullref{vowel-initial.epenthesis}, an epenthetic /ʔ/ has been inserted.
This is done because of restrictions surrounding diphthongs.
The full explanation can be found in \Sref{sec:phonotactics}.

For stems beginning with a syllable containing a diphthong, the reduplication surfaces a bit differently:

\pex
\a
\begingl
\glpreamble auna
\pronounced{ˈau.na}\endpreamble
Ø-auna[\textsc{abs-}moon]
\endgl

\a
\begingl
\glpreamble aʔauna
\pronounced{ˈa.ʔo.na}\endpreamble
aʔ\textasciitilde auna[\textsc{erg\textasciitilde}moon]
\endgl

\a
\begingl
\glpreamble meauna
\pronounced{ˈme.au.na}\endpreamble
<me>Ø-auna[\textsc{<pl>abs-}moon]
\endgl

\a
\begingl
\glpreamble ameauna
\pronounced{ˈa.mə.au.na}\endpreamble
a<me>auna[\textsc{erg<pl>\textasciitilde}moon]
\endgl
\xe
%TODO: Write a CVV-initial inflection example

\section{Indefinite Constructions}
\subsection{Indicative Mood}
Indefinite pronouns in particular don't exist as a separate class of words in \langname .
Rather, these sorts of constructions are created with interrogatives, and in certain situations leverage uncertainty evidentials, to impart the same meaning.

When responding to something where the answer would typically be an indefinite pronoun, a few strategies can be taken.
\subsubsection{Response}
Firstly, and often preferred, is the method of asking a question in return.
For example, if one were to ask ``Who touched that?'', the most simple and standard reply would be ``Who did?''.
In English this may be seen as sarcastic and possibly rude, but this is not the case with \langname .

If one were to reference \Sref{sec:certainty}, they may be wondering why these certainty particles are not more widely used in these scenarios.
The usages of these particles refers more to having an understanding (rather than strict knowing), and thus can only be used in certain scenarios.
More details about these particles in particular will be found in the corresponding section listed above.

This follows for any sort of nonpolar question.

The example exchange noted above has been reproduced in a glossed form to illustrate this pattern.

\pex<indef-pronominals>
\a<question>
\begingl
\glpreamble papane mutaʔali
\pronounced{ˈpa.pa.nə ˈmu.ta.ʔa.li}\endpreamble
pa\textasciitilde pane[erg\textasciitilde person]
mutaʔali[\textsc{pst.pfv:}touch\textsc{[q]}]
\glft `Who touched (that)?'
\endgl

\a<answer>
\begingl
\glpreamble pane
\pronounced{ˈpa.nə}\endpreamble
pane[person]
\glft `Someone'
\endgl
\xe

This short exchange actually illustrates several concepts simultaneously, many of which will be further addressed in later sections.

Firstly, in Example \getfullref{indef-pronominals.answer}, an utterance which would be typically translated as a simple interrogative `who?' is treated as an indefinite pronoun, which is an entirely grammatical response to this question.
This is also significant in that it is not a response to the question meant as a repair strategy.
These questions are constructed entirely different, and are detailed in \Sref{sec:repair_response}.

In regards to Example \getfullref{indef-pronominals.question}, it should be noted that demonstratives are often omitted for transitive verbs with 3\textsuperscript{rd} person agents, and the agent is inflected ergatively instead.

\subsubsection{Indefinite Agent}
If the agent of a verb would typically be an indefinite pronoun, a combination of peripherasis With additional verbs, and uncertainty evidentials is used.

\ex
\begingl
\glpreamble pane wu ehaixa i munihasi lu me malaku, me ɸai ʔike
\pronounced{ˈpa.ne ˈwu ˈe.hai.xa ˈi ˈmu.ni.ha.si ˈlu ˈme ˈma.la.ku | ˈme ˈɸai ˈʔi.kə}\endpreamble
\nogloss{\lbrack}
pa\textasciitilde pane[\textsc{erg\textasciitilde}person]
wu[on\_top]
ehaixa[\textsc{name}]
i[inside]
munihasi[\textsc{pst.pfv}:give\textsc{:inv}]
lu[\textsc{obv}]
me[\textsc{1sg}]
malaku[cat]
\nogloss{\rbrack}
me\textasciitilde me[\textsc{erg\textasciitilde 1sg}]
ɸai[\textsc{neg}]
ʔike[know]
\glft `(Someone) from Ehaixa gave me this cat, I don't know (them).'
\endgl
\xe

In this case, the construction with \langword{ʔike} is necessary to specify the exact agent is unknown.
Evidentials cannot be used here, because it is not the giving itself which is uncertain, and the evidentials modify verbs.

In this scenario, the verb phrase could also be placed into a relative clause, although this is often not particularly practical unless paired with \nameref{sec:ellipsis}.
\subsubsection{Indefinite Patient}
\subsection{Interrogative Mood}
Interrogative contexts are one of such situations in which the certainty evidentials will be utilized most often.
In these situations the evidentials are appropriate, because the speaker of the utterance is expressing uncertainty about whether an event has happened in regards to a particular person.

\ex
\begingl
\glpreamble tesune mukula pane
\pronounced{ˈte.su.nə ˈmu.kula ˈpa.nə}\endpreamble
tesune[\textsc{uncert}]
mukula[\textsc{pst.pfv:}hurt\textsc{[q]}]
pane[person]
\glft `Did you hurt someone?'
\endgl
\xe

In this sort of situation, evidentials are preferred.
This has to do with specifically which part of the utterance the speaker is unsure of.
Here, the speaker is unsure of whether the event (hurting) actually occurred or not, thus a construction which modifies the verb is sensible.
\langword{Pane} is simply here as a way of marking the 3\textsuperscript{rd} person, because if the object was completely omitted, a direct verb in this configuration would have a 1\textsuperscript{st} person object.

\section{Demonstrative Pronouns}
While \langname\ does not have distinct 3\textsuperscript{rd} person independent pronouns, it does possess demonstratives, which on occasion can fulfill the same basic role a pronoun might.
Notably, these demonstratives are fossilized such that role marking is done with the older \textsc{prox} and \textsc{obv} forms, while maintaining the infixed plural marking applied synchronically.
\begin{description}
  \item[Proximal:] kini
  \item[Medial:] nata
  \item[Distal:] atahau
\end{description}


These demonstratives are used not only as a way of referring to  objects, but also as a way of orienting other phrases differently.
These will be discussed in greater detail in \Pref{pt:syntax}.
Nonetheless, a few uses will be described here for the sake of timely introduction.

\subsubsection{With Adpositional Phrases}

\ex
\begingl
\glpreamble kini sa nuka
\pronounced{ˈki.ni ˈsa ˈnu.ka}\endpreamble
kini[\textsc{dem\_prox}]
sa[outside\_of]
nuka[glass]
\glft `[The] outside of the window'
\endgl
\xe

\ex
\begingl
\glpreamble nataʔe iku i nuka
\pronounced{ˈna.ta.ʔə ˈi.ku ˈi ˈnu.ka}\endpreamble
nata[\textsc{dem\_med}]
ʔe[\textsc{assoc}]
iku[old]
i[inside\_of]
nuka[glass]
\glft `That window is weak'
\endgl
\xe
