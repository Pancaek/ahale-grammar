\chapter{Nominal}\label{ch:morpho-nom}
\section{Inflectional Morphology}\label{sec:morpho-nom-inf}
Inflectional morphology is very limited for nouns. The most marked nouns only inflect for case marking and plurality. However, depending on the context, plurality is optional. Depending on the constructions used, inflection of the noun may be even ungrammatical, where in other situations the same inflection would be entirely sensible.

\subsection{Case Marking}\label{sec:case-marking}
\langname 's case system is incredibly small. it consists solely of the cases used for morphosyntactic purposes; the ergative and absolutive cases. The ergative case is marked with reduplication of the initial syllable, while the absolutive case remains unmarked. This becomes slightly less transparent when this interacts with phonotactics, and when this reduplication triggers allophony.

\aside{\fleuron\ While phonological form is not the main focus of this section, the variation that can occur is such that I find it useful. It \textit{is} true that much of the prose simply reiterates what is found in the pronunciation information found in the gloss, however I consider this approach more useful for grasping the reasoning and triggers behind these phonological changes.}

With \notabletext{CV-initial forms,} this is quite straightforward:

\begin{subexamples}
  \baarucols{2}
  \ex
    \preamble pane
    \pronunciation ˈpa.nə
    \gloss
      pane & person[ABS] \\
  \ex
    \preamble papane
    \pronunciation pa.ˈpa.nə
    \gloss
      pa\allo pane & ERG\allo person \\
\end{subexamples}

\notabletext{V-initial forms} are slightly different, as adjacent vowels cannot be identical. An epenthetic \phomtext{ʔ} is inserted to avoid this.

\begin{subexamples}
  \baarucols{2}
  \ex
    \preamble ana
    \pronunciation ˈa.na
    \gloss
      ana & eye[ABS] \\
  \ex
    \preamble a'ana
    \pronunciation aʔ.ˈa.na
    \gloss
      aʔ\allo ana & ERG\allo eye \\
\end{subexamples}

\begin{subexamples}
  \baarucols{2}
  \ex
    \preamble ele
    \pronunciation ə.ˈle
    \gloss
      ele & heaven[ABS] \\
  \ex
    \preamble e'ele
    \pronunciation əʔ.ə.ˈle
    \gloss
      eʔ\allo ele & ERG\allo heaven \\
\end{subexamples}

\phomtext{ə} initial forms (such as the one above) are particularly interesting, in that they most clearly show an interaction between this process of reduplication and allophony. This reduplication can be described with the morpheme \morphtext{V,}. As a result \phomtext{əʔ.ə.lə} is phonetically \phontext{əʔ.ə.ˈle}, rather than \phontext{əʔ.ˈe.lə} as it would be with a standard morpheme.

\notabletext{CVV-initial forms} undergo a slightly different process. The reduplication in this situation does not even apply to the entire syllable, but rather only to the CV segment.

\begin{subexamples}
  \baarucols{2}
  \ex
    \preamble keuha
    \pronunciation ˈko.ha
    \gloss
      keuha & leaf[ABS] \\
  \ex
    \preamble kekeuha
    \pronunciation kə.ˈko.ha
    \gloss
      ke\allo keuha & ERG\allo leaf \\
\end{subexamples}

Consequently for LSA, the reduplicated vowel may be different than its phonetic realization in the stem due to an underlying diphthong.

\notabletext{VV-initial forms} work similarly to CVV forms, with the addition of an epenthetic \phomtext{ʔ} at the morpheme boundary:

\begin{subexamples}
  \baarucols{2}
  \ex
    \preamble auna
    \pronunciation ˈau.na
    \gloss
      auna & moon[ABS] \\
  \ex
    \preamble a'auna
    \pronunciation aʔ.ˈo.na
    \gloss
      aʔ\allo auna & ERG\allo moon \\
\end{subexamples}

As illustrated by the previous set of examples, this also triggers mono\-phthongization of the stem's initial syllable.

\subsection{Pluralization}\label{sec:pluralization}
Pluralization is very regularly marked through the infixation of \morphtext{me} between the reduplicated syllable of an ergative form and the stem. In absolutive forms, this gives the appearance that \morphtext{me} is a prefix.

Let's revisit some of the previous examples, now given in their plural forms:

\begin{subexamples}
  \baarucols{2}
  \ex
    \preamble mepane
    \pronunciation mə.ˈpa.nə
    \gloss
      <me>pane & <PL>person[ABS] \\
  \ex
    \preamble pamepane
    \pronunciation pa.ˈme.pa.nə
    \gloss
      pa\allo <me>pane & ERG\allo <PL>person \\
\end{subexamples}

\begin{subexamples}
  \baarucols{2}
  \ex
    \preamble meana
    \pronunciation mə.ˈa.na
    \gloss
      <me>ana & <PL>eye[ABS] \\
  \ex
    \preamble ameana
    \pronunciation a.ˈme.a.na
    \gloss
      a\allo <me>ana & ERG\allo <PL>eye \\
\end{subexamples}


\begin{subexamples}
  \label{ex:double-schwa}
  \baarucols{2}
  \ex
  \preamble meele
  \pronunciation ˈme.lə
  \gloss
    <me>ele & <PL>heaven[ABS] \\
  \ex
    \label{ex:double-schwa-erg}
    \preamble emeele
    \pronunciation ə.ˈme.lə
    \gloss
      e\allo <me>ele & ERG\allo <PL>heaven \\
\end{subexamples}

\aside{Respellings of loanwords from other languages commonly utilize \orthotext{ee}, although older generations generally dislike this emerging tendency, as some native words are being supplanted by these newer loans.}

\begin{subexamples}
  \baarucols{2}
  \ex
    \preamble mekeuha
    \pronunciation mə.ˈko.ha
    \gloss
      <me>keuha & <PL>leaf[ABS] \\
  \ex
    \preamble kemekeuha
    \pronunciation kə.ˈme.ko.ha
    \gloss
      ke\allo ke<me>uha & ERG\allo <PL>leaf \\
\end{subexamples}

\begin{subexamples}
  \baarucols{2}
  \ex
    \preamble meauna
    \pronunciation mə.ˈau.na
    \gloss
      <me>auna & <PL>moon[ABS] \\
  \ex
    \preamble ameauna
    \pronunciation a.ˈme.o.na
    \gloss
      a\allo <me>auna & ERG\allo <PL>moon \\
    \end{subexamples}

\baaruref{ex:double-schwa} is notable for being one of the few situations utilizing \orthotext{ee}. it is preserved in some words to distinguish otherwise opaque stress or as a result of etymology such as in \baaruref{ex:double-schwa-erg}.

\section{Derivational Morphology}
\begin{description}
  \item[-la] a noun becomes an adjective \transtext{composed from X}
  \item[-ku] a noun becomes a verb with the meaning \nativetext{ewi-siha hasi X ta} \transtext{to happen as if by X, to appear as X}
\end{description}

\subsection{-la}

Composition is the primary purpose of \nativetext{-la}, being fairly straightforward:

\baabbrev{adjz}{adjectivizer}
\begin{examples}
  \baarucols{2}
  \ex
    \gloss
      lelu & spoon \\
      suxe & \MC2 processed\_wood-ADJZ \\
      -la & \\
    \tr wooden spoon
  \ex
    \preamble kele wasila
    \pronunciation kə.ˈle ˈwa.si.la
    \gloss
    kele & bowl \\
    wasi & river\_stone-ADJZ \MC2 \\
    -la & \\
    \tr stone bowl
  \end{examples}

  However, this isn't it's only meaning. It can also express the concept of origin, in the context of people and ideas. Crucially though, this type of origin refers to a much more generic beginning, rather than in reference to a single event.

  A key example of this difference can be demonstrated by contrasting two different strategies of expressing residency.
  \baabbrev[\FIRST]{1}{first person}
  \baabbrev{name}{name}
  \begin{examples}
    \baarucols{2}
    \ex
      \preamble me ɸene wu ehaixa
      \pronunciation ˈme ɸə.ˈne ˈwu ə.ˈhai.xa
      \gloss
        me & 1sg \\
        ɸene & live \\
        wu & on\_top \\
        Ehaixa & NAME \\
      \tr I live in Ehaixa.
    \ex
      \preamble me ehaixala
      \pronunciation ˈme ə.ˈhai.xa.la
      \gloss
        me & 1sg \\
        Ehaixa & NAME-ADJZ \MC2 \\
        -la & \\
      \tr I am from Ehaixa.
\end{examples}
