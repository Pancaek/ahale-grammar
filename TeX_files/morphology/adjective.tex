\chapter{Adjectival}\label{ch:adjectives}
Adjectives are not a unique class of words in \langname . What may look like ``adjectives'' on the surface are simply nouns.

\ex
\begingl
\glpreamble masaʔe si sixi
\pronounced{ˈma.sa.ʔə ˈsi ˈsi.xi}\endpreamble
\nogloss{\lbrack}
masa[sun]
ʔe[\textsc{assoc}]
si[brightness]
\nogloss{\rbrack}
sixi[\textsc{npst.ipfv}-shine]
\glft `The bright sun shines.'
\endgl
\xe

If the noun being modified in this way has ergative marking, it should be noted that the noun \textit{does not} inflect in agreement with the main noun.

\section{Adjective Ordering}

\langname 's\ basic adjective ordering is: «opinion» «size» «physical quality» «shape» «age» «color» «origin» «material» «type» «purpose»

In some cases however, this basic ordering may be deviated from. A single adjective may be placed before \langword{ʔe,} allowing the main noun itself to be dropped, and the main to be referenced in futher discourse using the promoted adjective + ʔe as a logophoric pronoun. This is particularly useful when many of the same object with similar but differing qualities are being discussed for extended lengths of time (for example, a discussion about two different people, or about several types of a similar object). It may also be used, as seen below to chain clauses together.

\ex
\begingl
\glpreamble masa siʔe sixi siʔe sihu kaʔa
\pronounced{ˈma.sa ˈsi.ʔe ˈsi.xi ˈsi.ʔe ˈsi.hu ˈka.ʔa}\endpreamble
\nogloss{\lbrack}
masa[\textsc{sun}]
si[brightness]
ʔe[\textsc{assoc}]
\nogloss{\rbrack}
sixi[\textsc{npst.ipfv}-shine]
si-ʔe[brightness-ʔe]
sihu[happen]
kaʔa[happiness]
\glft `The bright sun shines, and its light makes me happy.'
\endgl
\xe

%TODO: Cover examples with many adjectives 