\chapter{Adjectival}\label{ch:adjectives}
Adjectives are not a unique class of words in \langname . What may look like ``adjectives'' on the surface are simply nouns.

\ex
\begingl
\glpreamble masaʔe si sixi
\pronounced{ˈma.sa.ʔə ˈsi ˈsi.xi}\endpreamble
\nogloss{\lbrack}
masa[sun]
ʔe[\textsc{assoc}]
si[brightness]
\nogloss{\rbrack}
sixi[\textsc{npst.ipfv}-shine]
\glft `The bright sun shines.'
\endgl
\xe

If the noun being modified in this way has ergative marking, it should be noted that the noun \textit{does not} inflect in agreement with the main noun.

\section{Adjective Ordering}

\langname 's\ basic adjective ordering is: «opinion» «size» «physical quality» «shape» «age» «color» «origin» «material» «type» «purpose»

\section{NP-Ellipsis with \langword{ʔe} (\langword{ʔe} Ellipsis)}

In some cases however, this basic ordering may be deviated from. A single adjective may be placed before \langword{ʔe,} allowing the main noun itself to be dropped, and the main to be referenced in futher discourse using the promoted adjective + ʔe in a form of NP-ellipsis. This is particularly useful when many of the same object with similar but differing qualities are being discussed for extended lengths of time (for example, a discussion about two different people, or about several types of a similar object). It may also be used, as seen below to chain clauses together.

This ellipsis is not entirely dissimilar to the function of ``That X thing'', though the construction itself is quite different.


\ex<optional_ellipsis>
\begingl
\glpreamble masa siʔe sixi, siʔe kaʔa
\pronounced{ˈma.sa ˈsi.ʔe ˈsi.xi | ˈsi.ʔe ˈka.ʔa}\endpreamble
\nogloss{\lbrack}
masa[sun]
si[brightness]
ʔe[\textsc{assoc}]
\nogloss{\rbrack}
sixi[\textsc{npst.ipfv}-shine]
si-ʔe[brightness-ʔe]
kaʔa[happiness]
\glft `The bright sun shines, and its light makes me happy.'
\endgl
\xe

It should also be noted that this ellipsis is often not the only way sentences which utilize it could be expressed. In cases such as that of Example \getfullref{optional_ellipsis}, ellipsis is purely a stylistic preference, especially when used isolated from other context. In particular, \getfullref{optional_ellipsis} may also be expressed as:

\ex
\begingl
\glpreamble masaʔe si sixi, siʔe kaʔa
\pronounced{ˈma.sa.ʔe ˈsi ˈsi.xi | ˈta ˈka.ʔa}\endpreamble
\nogloss{\lbrack}
masa[\textsc{sun}]
ʔe[\textsc{assoc}]
si[brightness]
\nogloss{\rbrack}
sixi[\textsc{npst.ipfv}-shine]
ta[\textsc{prox}]
kaʔa[happiness]
\glft `The bright sun shines, and its light makes me happy.'
\endgl
\xe

Here, the same information is presented, but in a slightly different manner. \textit{ʔe-ellipsis,} as it will now be referred to as, is not utilized. Default adjective order is returned to, and the proximate \langword{ta} is used resumptively in reference to \langword{masa.}

The necessity of \langword{ta} here is dependent on the speaker, as \langword{sixi} is strictly intransitive and the ommission of \langword{ta} may not impact intelligibility for this group of speakers, where context is suffucuent in maintaining understanding.
%TODO: Cover examples with many adjectives