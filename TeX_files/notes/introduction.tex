\chapter{Introduction}
\section{Document structure}
\subsection{Prose}
All prose within the main block of text is self-contained, such that reading it on its own is sufficient to a gain basic understanding of the material.
Other details which accompany the main text support and expand its content from several different prespectives.
The additional materials are designed this way not to discourage deeper reading, but rather to keep the most important details easily accessible without sacrificing depth of coverage for nuanced topics.

A variety of formatting standards are employed such that important details are easily and consistently identifiable:

\begin{itemize}
  \item Particularly important parts of an explanation, such as those which address a \detail{common misconception} or \detail{distinction between constructions} are italicized.
  \item Notable terms are colored \highlight{like this} when introduced, and may be colored in following instances to further emphasize them.
\end{itemize}

When text is written in \langname\ \detail{alongside English prose}:

\begin{itemize}
  \item Romanization is italicized.
  \item Native orthography is bolded, or surrounded by parentheses if a native term is used as an alternative to an English description.
  \item Inline translations are surrounded by single quotes.
\end{itemize}

A maximal example is given in following excerpt:

\begin{quote}
 The word \rom{awale} \native{接} \trans{to touch} is often used to demonstrate a particular process of intensifying reduplication most often applied to verbs of sensation. This process \nativeparen{続纘}\notetoself{This says \rom{mi'i hasi} which is roughly \trans{to do over and over again}, this itself being a sort of pun on \rom{hasi} meaning \trans{continue}, but more in relation to effort and consistency. An alternate translation of this, \trans{to work at} is also perhaps useful.} often imparts connotations of anger or overexertion, but can be used playfully to describe achievement in spite of circumstances.
\end{quote}

\subsection{Footnotes}
Footnotes are used to convey external observations, as well as provide linguistic analyses of particular details.
They may provide additional glossed examples, or discuss design choices in regards to a particular construction.
Also, they may simply be used when describing nuance in a manner unlike that of a native speaker.

\subsection{Margin notes}
Margin notes are used when describing details from a native speaker's perspective.
These are most frequently used when discussing word choice and similar nuance, containing a larger proportion of native terminology in their descriptions.
