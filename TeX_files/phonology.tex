\part{Phonology}\label{prt:phonology}
\chapter{Segmental Phonology}\label{ch:phono-seg}
\section{Inventory}\label{sec:phono-inv}

\begin{table}[ht]
  \centering
  \begin{tabular}{*{7}{c}}
    \toprule
    & Labial & Alveolar & Velar & Glottal \\\midrule
    Plosive   & p      & t        & k     & ʔ       \\
    Nasal     & m      & n        &       &         \\
    Fricative & ɸ      & s        & x     & h       \\
    Sonorant  & w      & l        &       &         \\
    \bottomrule
  \end{tabular}
  \caption{Consonant Inventory}
  \label{table:consonants}
\end{table}

\aside{\fleuron\ \phontext{e} is an allophone of \phomtext{ə} rather than a phoneme unto itself, as it may first appear (see \REL\ and \ASSOC\ \nativetext{ʔe})
}

\begin{figure}[ht]
  \centering
  \begin{vowel}
    \vpoint{0}{3}{i}
    \vpoint{2}{3}{u}
    \vpoint{1}{1.5}{ə}
    \vpoint{1}{0}{a}
    \varrow{a}{u}
    \varrow{a}{i}
  \end{vowel}
  \caption{Phonemic Vowel Inventory}
  \label{table:vowel_phonemes}
\end{figure}

Diphthongs consisting of \phomtext{ə} + high vowel can be observed in a few words, however these diphthongs are prone to collapse. In all but the most conservative of dialects, these are realized as \phomtext{əi əu} \phontext{e o}.

\vfill %! This is not smart don't do this
\section{Stress}\label{sec:phono-stress}
\subsection{Assignment}\label{sec:phono-stress-assign}
Stress is generally assigned to the initial syllable of a root. However, at this stage, \phomtext{ə} will not generally receive stress, which moves to the second syllable instead (even if that syllable contains a schwa as well).

A subset of affixes (including those formed with reduplication) are considered ``weak'' in that, like schwa, they repel stress.

Such morphemes will be marked with a comma in morphemic transcription, rather than the usual period used at syllable boundaries. Thus \morphtext{ʔə,} refers to a morpheme \phomtext{ʔə} which repels stress.

\subsection{Realization}
Under most circumstances, lexical stress is realized as a slight rise in pitch on the affected syllable, which then gradually falls back to ``standard'' pitch across the remaining part of the word. Because stress is very strongly initial, this means that generally, pitch will fall gradually across words and phrases.

This is of course also affected by phrase- and sentence-level prosody, which will be discussed in \Cref{ch:suprasegmental-phonology}.

\section{Allophony}\label{sec:phono-allo}
This section will be structured a bit differently. Because allophony and diachronic processes in general occur in a particular order, larger processes will constitute their own sections, as opposed to those more easily described in standard notation.

While this document will cover some dialectical differences, examples given which include phonetic information will be in reference to a standard dialect, \notabletext{Literary Standard \langname}, Which will be abbreviated \notabletext{LSA}.

\begin{enumerate}
  \item {
    Stressed \phomtext{ə} becomes \phontext{e}

    \phon{ə\phonfeat{+stress}}{e}
    }
  \item \titleref{sec:phono-allo-lengthening}
  \item {
    \phontext{ə} is elided between homorganic consonants

    \phonc{ə}{Ø}{\phonfeat{αcplace} \phold \phonfeat{αcplace}}
    }
    \item {
      \phontext{n} + plosive clusters cause place assimilation of the nasal

      \phonc{\phonfeat{+nasal}}{\phonfeat{αcplace}}{\phold \phonfeat{-cont}}
  }
    \item {
      Nasal + plosive clusters assimilate

      \phonc{\phonfeat{-cont}}{\phonfeat{+nasal}}{\phonfeat{+nasal} \phold}
  }

  \item {
      Unstressed final vowels are elided

      \phonc{V\phonfeat{-stress}}{Ø}{\phold \$}
  }
  % \aside{\fleuron\ If I'm being completely honest, I have no idea why the process of affrication is done with +delayed release. I'm simply following the conventions as outlined by \url{http://www.artoflanguageinvention.com/papers/features.pdf}}
  % \item {
  %     Word final geminated plosives are affricated

  %     \phonc{C\phonfeat{-cont \\ +long}}{C\phonfeat{+del release\footnotemark \\ -long}}{\phold \$}
  % }

\end{enumerate}

\subsection{Prosodic Lengthening}\label{sec:phono-allo-lengthening}
In some situations, syllable shape can cause gemination of the surrounding syllables --- prosodic lengthening --- to allow for more consistent syllable timing. The most notable of these situations is when a CV syllable with a plosive initial comes after a non-plosive CVV sequence\footnotemark.

\footnotetext{This is described as a CVV \textit{sequence} rather than a CVV syllable, because this process is not dependent on \phontext{VV} being a true diphthong. This can be seen in the first example, where \phontext{ua} is not a diphthong, but still triggers this change.}

Thus, the following hypothetical pairs may emerge:
\begin{itemize}
  \item \phomtext{suaki} \phontext{ˈsu.a.kːi} and \phomtext{suki} \phontext{ˈsu.ki}
  \item \phomtext{maipi} \phontext{ˈmai.pːi} and \phomtext{mapi} \phontext{ˈma.pi}
  \item \phomtext{iwəiku} \phontext{ˈi.we.kːu}\footnotemark\ and \phomtext{iwəku} \phontext{ˈi.wə.ku}
\end{itemize}

\footnotetext{This more aggressive simplification of diphthongs may give the appearance that plosives may geminate without an apparent trigger. Of course, this only applies to dialects which do not preserve the original diphthongs.}

\aside{\fleuron\ While the moraic analysis of this process might seem unconvential, one might posit that this process is triggered by a heavy syllable followed by a light one, and that CVV is heavy, while VV is not.}
One thing of note is that the onset of the CVV syllable is mandatory to trigger this process. Observe the following hypothetical forms which do not undergo this change due to their VVCV shape:
\begin{itemize}
  \item \phomtext{auku} \phontext{ˈau.ku}
  \item \phomtext{aiʔi} \phontext{ˈai.ʔi}
  \item \phomtext{euti} \phontext{ˈo.ti}
\end{itemize}

\chapter{Suprasegmental Phonology}\label{ch:suprasegmental-phonology}
