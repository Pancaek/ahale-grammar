\part{Phonology}\label{prt:phonology}
\chapter{Segmental Phonology}\label{ch:seg-phono}
\section{Inventory}\label{sec:phono-inv}

\begin{table}[ht]
  \centering
  \begin{tabular}{*{7}{c}}
    \toprule
    & Labial & Alveolar & Velar & Glottal \\\midrule
    Plosive   & p      & t        & k     & ʔ       \\
    Nasal     & m      & n        &       &         \\
    Fricative & ɸ      & s        & x     & h       \\
    Sonorant  & w      & l        &       &         \\
    \bottomrule
  \end{tabular}
  \caption{Consonant Inventory}
  \label{table:consonants}
\end{table}

\aside{\fleuron \phontext{e} is an allophone of \phomtext{ə}, rather than a phoneme unto itself as it may first appear, (see \textsc{rel} and \textsc{assoc} \nativetext{ʔe})
}

\begin{figure}[ht]
  \centering
  \begin{vowel}
    \vpoint{0}{0}{i}
    \vpoint{0}{2}{u}
    \vpoint{1.5}{1}{ə}
    \vpoint{3}{1}{a}
    \varrow{a}{u}
    \varrow{a}{i}
  \end{vowel}
  \caption{Phonemic Vowel Inventory}
  \label{table:vowel_phonemes}
\end{figure}

Diphthongs consisting of \phomtext{ə} + high vowel can be observed in a few words, however these diphthongs are prone to collapse. In all but the most conservative of dialects, these are realized as \phomtext{əi əu} \phontext{e o}.

\vfill %! This is not smart don't do this
\section{Stress}
\subsection{Assignment}
Stress is generally assigned to the initial syllable of a root. However, \phomtext{ə} generally will not recieve stress at this stage and move to the second syllable (even if this is also a schwa).

A subset of affixes (including, but not limited to those which utilize reduplication) are considered ``weak'', in that, like schwa, they repel stress.

Such morphemes will be marked with a comma in morphemic transcription, rather than the usual period used at syllable boundaries. Thus \morphtext{ʔə,} refers to a morpheme \phomtext{ʔə} which repels stress.

\subsection{Realization}
Under most circumstances, lexical stress is realized as a slight rise in pitch on the affected syllable, which then gradually falls back to ``standard'' pitch across the remaining part of the word. Because stress is very strongly initial, this means that generally, pitch will fall gradually across words and phrases.

This is of course also affected by phrase- and sentence-level prosody, which will be discussed in \Cref{ch:suprasegmental-phonology}.

\section{Allophony}
This section will be structured a bit differently from the others encountered thus far. Because allophony and diachronic processes in general occur in a particular order, Sectioning will be used to Separate larger processes from those more easily described in standard notation.

The ordering of rules will begin below, and these larger processes will be contained within subsections which will be referenced in the list.

\begin{enumerate}
  \item {
    Stressed \phomtext{ə} becomes \phontext{e}

    \phon{ə\phonfeat{+stress}}{e}
    }
  \item \titleref{sec:allophony-lengthening}
  \item {
    \phontext{ə} is elided between homorganic consonants

    \phonc{ə}{Ø}{\phonfeat{αcplace} \phold \phonfeat{αcplace}}
    }
    \item {
      \phontext{n} + plosive clusters cause place assimilation of the nasal

      \phonc{\phonfeat{+nasal}}{\phonfeat{αcplace}}{\phold \phonfeat{-cont}}
  }
    \item {
      Nasal + plosive clusters assimilate

      \phonc{\phonfeat{-cont}}{\phonfeat{+nasal}}{\phonfeat{+nasal} \phold}
  }

  \item {
      Unstressed final vowels are elided

      \phonc{V\phonfeat{-stress}}{Ø}{\phold \$}
  }
  % \aside{\fleuron\ If I'm being completely honest, I have no idea why the process of affrication is done with +delayed release. I'm simply following the conventions as outlined by \url{http://www.artoflanguageinvention.com/papers/features.pdf}}
  % \item {
  %     Word final geminated plosives are affricated

  %     \phonc{C\phonfeat{-cont \\ +long}}{C\phonfeat{+del release\footnotemark \\ -long}}{\phold \$}
  % }

\end{enumerate}

\subsection{Prosodic Lengthening}\label{sec:allophony-lengthening}
In some situations, syllable shape can cause gemination of the surrounding syllables--- prosodic lengthening, if you will, to allow for more consistent syllable-timing. The most notable of these situations is when a CV syllable, where C is a plosive, comes after a CVV sequence\footnotemark\ beginning in a non-plosive consonant.

\footnotetext{This is described as a CVV \textit{sequence} rather than a CVV syllable, because this process is not dependent on \phontext{VV} being a true diphthong. This can be seen in the first example, where \phontext{ua} Is not a diphthong but still triggers this change.}

Thus, pairs such as the following hypothetical forms may emerge:
\begin{itemize}
  \item \phomtext{suaki} \phontext{ˈsu.a.kːi} and \phomtext{suki} \phontext{ˈsu.ki}
  \item \phomtext{maipi} \phontext{ˈmai.pːi} and \phomtext{mapi} \phontext{ˈma.pi}
  \item \phomtext{iwəiku} \phontext{ˈi.we.kːu}\footnotemark\ and \phomtext{iwəku} \phontext{ˈi.wə.ku}
\end{itemize}

\footnotetext{This more aggressive simplification of diphthongs may give the appearance that plosives may geminate without an apparent trigger. Of course, this only applies to dialects which do not preserve the original diphthongs.}

\aside{\fleuron\ While moraic analysis of this process is not typical, one might suppose that this process is triggered by a heavy syllable followed by a light one, and that CVV is heavy, while VV is not.}
One thing of note is that the onset of the CVV syllable is mandatory to trigger this process. Observe the following hypothetical forms, Which do not undergo this change due to their VVCV shape:
\begin{itemize}
  \item \phomtext{auku} \phontext{ˈau.ku}
  \item \phomtext{aiʔi} \phontext{ˈai.ʔi}
  \item \phomtext{euti} \phontext{ˈo.ti}
\end{itemize}


\chapter{Suprasegmental Phonology}\label{ch:suprasegmental-phonology}
