\chapter{Phonology}

\section{Phonemic Inventory}
\begin{table}[ht]
  \centering
  \begin{tabular}{*{7}{c}}
    \toprule
              & Labial & Alveolar & Velar & Glottal \\\midrule
    Plosive   & p      & t        & k     & ʔ       \\
    Nasal     & m      & n        &       &         \\
    Fricative & ɸ      & s        & x     & h       \\
    Sonorant  & w      & l        &       &         \\
    \bottomrule
  \end{tabular}
  \caption{Consonant Inventory}
  \label{table:consonants}
\end{table}

%TODO: Add vowel realization blobs

\begin{figure}[ht]
  \centering
  \begin{vowel}
    \vpoint{0}{0}{i}
    \vpoint{0}{2}{u}
    \vpoint{1.5}{1}{ə}
    \vpoint{3}{1}{a}
    \varrow{a}{u}
    \varrow{a}{i}
  \end{vowel}
  \caption{Phonemic Vowels}
  \label{table:vowel_phonemes}
\end{figure}

% \begin{figure}[ht]
%   \centering
%   \begin{vowel}
%     \vpoint{0}{0}{i}
%     \vpoint{0}{2}{u}
%     \vpoint{1.5}{1}{ə}
%     \vpoint{3}{1}{a}
%   \end{vowel}
%   \caption{Phonetic Vowel Ranges}
%   \label{table:vowel_phones}
% \end{figure}

\section{Phonotactics}\label{sec:phonotactics}

(C)V(V)
\begin{description}
  \item[C:] Any consonant
  \item[V:] Any vowel
\end{description}

Diphthongs are permitted, provided the vowels differ in height. Under this rule, the diphthongs /iu/ and /ui/ are disallowed. The similar sequences /iʔu/ and /uʔi/, however, are permitted.

\section{Stress}
Stress is typically placed on the first syllable in a word. The only exception is when a word begins with /ə/, in which case it is placed on the second syllable. Stressed /ə/ is phonologically unstable, which makes it susceptible to shifting elsewhere. This can be observed in the following hypothetical forms: /ə.pa.lo/ [əˈpalo], and /pa.lo/ [ˈpalo]. Note how both forms are stressed on /pa/, even though the syllable is positioned differently in the word.

\section{Allophony}

%TODO: Add more to this section

/h/ realized as [ɣ] in stressed syllables

\phonc{h}{ɣ}{\phold V\phonfeat{+stress}}

/ɘ/ is realized as [e] in stressed syllables

\phonc{ə}{e}{\phold\phonfeat{+stress}}

/ɸ/ and /h/ are in free variation before /u/

/au/ monophthongizes to [o] after glottal consonants

\phonc{au}{o}{C\phonfeat{+glottal}\phold}
