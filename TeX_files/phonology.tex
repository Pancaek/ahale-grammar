\part{Phonology}

\chapter{Segmental Phonology}
\section{Inventory}

\begin{table}[ht]
  \centering
  \begin{tabular}{*{7}{c}}
    \toprule
    & Labial & Alveolar & Velar & Glottal \\\midrule
    Plosive   & p      & t        & k     & ʔ       \\
    Nasal     & m      & n        &       &         \\
    Fricative & ɸ      & s        & x     & h       \\
    Sonorant  & w      & l        &       &         \\
    \bottomrule
  \end{tabular}
  \caption{Consonant Inventory}
  \label{table:consonants}
\end{table}

\aside{\fleuron \phontext{e} is an allophone of \phomtext{ə}, rather than a phoneme unto itself as it may first appear, (see \textsc{rel} and \textsc{assoc} \nativetext{ʔe})
}

\begin{figure}[ht]
  \centering
  \begin{vowel}
    \vpoint{0}{0}{i}
    \vpoint{0}{2}{u}
    \vpoint{1.5}{1}{ə}
    \vpoint{3}{1}{a}
    \varrow{a}{u}
    \varrow{a}{i}
  \end{vowel}
  \caption{Phonemic Vowel Inventory}
  \label{table:vowel_phonemes}
\end{figure}

Diphthongs consisting of \phomtext{ə} + high vowel can be observed in a few words, however these diphthongs are prone to collapse. In all but the most conservative of dialects, these are realized as \phomtext{əi əu} \phontext{e o}.

% \vfill %! This is not smart don't do this
\section{Stress}
\subsection{Assignment}
Stress is generally assigned to the initial syllable of a root. However, \phomtext{ə} generally will not recieve stress at this stage and move to the second syllable (even if this is also a schwa).

A subset of affixes (including, but not limited to those which utilize reduplication) are considered ``weak'', in that, like schwa, they repel stress.

Such morphemes will be marked with a comma in morphemic transcription, rather than the usual period used at syllable boundaries. Thus \morphtext{ʔə,} refers to a morpheme \phomtext{ʔə} which repels stress.

\subsection{Realization}

\section{Allophony}
Stressed \phomtext{ə} becomes \phontext{e}

\phonc{ə}{e}{\phold\phonfeat{+stress}}

\chapter{Suprasegmental Phonology}
