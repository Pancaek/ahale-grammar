\chapter{Introduction}\label{ch:introduction}
\section{Conventions}\label{sec:conventions}
In this document, italic text will be used for native \langname\ words, unless enclosed in angle brackets (\nativetext{ahale} or \orthotext{ahale}, but never \orthotext{\nativetext{ahale}}).
\aside{This aside is from an internal perspective. It may discuss things such as word choice, or the cultural significance of phrases.}
\aside{\fleuron\ This aside is from an external perspective. It may discuss design choices or other such self-imposed challenges.}
The former is found mostly in running prose, and the latter in places where a word is being used as a more self-contained example, or simply if orthography is of particular interest.

\phomtext{forward slashes} are used for phonemic transcriptions, while \phontext{square brackets} are used for phonetic transcriptions.

When words are exemplified within prose, they are often followed by a short translation. These will be written surrounded by \transtext{single quotes}.

Notable terms will be written \notabletext{like this.}

``Double quotes'' are used for non-standard, ironic, or otherwise deviant use of terms.

\subsection{Document Structure}
Most of the content contained within this grammar will be found in the main prose. However, footnotes along with margin notes will be used for tangential or, in the case of footnotes, explanatory details about particular terms.

Margin notes will be reserved for explanation of decisions made in the various translations, or in the usage details of a specific construction.
In the case of notes from an external perspective, being preceded by a fleuron (\fleuron).

\subsection{Glossing}\label{sec:conventions-gloss}
Glosses are generally constructed as follows:

\begin{example}
  \lect Dialect
  \preamble Romanization (undelimited)
  \pronunciation pronunciation
  \gloss
  morphemic & morphemic \\
  transcription & transcription \\
  (object language) & (metalanguage) \\
  \tr Translation
  \source Source
  \end{example}

\section{History}\label{sec:history}
\subsection{Internal}\label{sec:hist-int}

\subsection{External}\label{sec:hist-ext}
\langname\ is the spiritual successor to my first conlang, Wei. As with many first attempts at conlanging, I consider Wei to be poorly made, and even more poorly described. Because of this, I consider \langname\ a successor only on account of my initial intentions.

Wei was very anglocentric, only a few steps away from a full relex, if even that. After I pronounced Wei abandoned in September 2019, I took a long break, and used that time to learn more about linguistics so that when I returned, I felt at least somewhat more prepared to make something I would be proud of.

When I began work on \langname , I wanted to stray as far away from my original mistakes as I could do comfortably. With that in mind, I outlined just a few goals for myself:

\begin{itemize}
  \item A morphosyntactic alignment other than nominative-accusative
  \item A minimalistic phonology which I would still be comfortable pronouncing
  \item Relatively free word order
  \item Some degree of non-concatenative morphology
  \item Thorough documentation, with many examples to explain nuances in translation (another goal of mine)
  \item Generally be interesting, fun, and fulfilling to work on
\end{itemize}

As the documentation currently stands, I feel I have achieved, and continue to work to achieve many of these goals. Depth of documentation, and nuance of lexical entries is something I'm still working on. Throughout the various iterations of this document, I hope that I can continue to improve on these areas to an even greater extent than I have at the time of writing this section.\footnotemark

\footnotetext{Last updated \DTMdisplaydate{2021}{1}{31}{-1}
}
\subsubsection{Typological Details}
While I don't find typology a terribly informative tool in regards to how language \textit{actually} functions, it can serve as an overview of concepts which will be encountered. As such, here's (another) short list of bullet points:

\begin{itemize}
  \item Utilizing a direct-inverse verbal system (resulting in a syntactically \textit{hierarchical} alignment)
  \item Ergative-absolutive morphology
  \item Strong pro-drop tendencies
  \item (A lot of) reduplication
  \item Use of verb serialization
\end{itemize}

In essence, the things which I find interesting.
