\chapter{Introduction}\label{ch:introduction}
\section{Conventions}\label{sec:conventions}
In this document, either italic text or angle brackets will be used for native \langname\ words (but not both -- \nativetext{ahale} or \orthotext{ahale}, but never \orthotext{\nativetext{ahale}}).
The former will be found mostly in running prose, whereas the latter will be reserved for more self-contained examples, or simply if orthography is of particular interest.

\phomtext{forward slashes} will denote phonemic transcription, and \phontext{square \\brackets} phonetic transcription.

When exemplifying words within prose, natural translations are prone to follow. These will then be written surrounded by \transtext{single quotes}.

Notable terms will be written \notabletext{like this.}

``Double quotes'' will be used for non-standard, ironic, or otherwise deviant use of terms.

\subsection{Document Structure}
Most of the content contained within this grammar will be found in the main prose. However, footnotes along with margin notes will be used for tangential or, in the case of footnotes, explanatory details about particular terms.
\aside{This aside is from an internal perspective. It may discuss word choice or the cultural significance of phrases.}%
\aside{\fleuron\ This aside is from an external perspective. It may discuss design choices or other such self-imposed challenges.}%

Margin notes will serve as explanation for decisions made in the various translations, or in the usage details of a specific construction.
Notes from an external perspective will be preceded by a fleuron (\fleuron).

\subsection{Glossing}\label{sec:conventions-gloss}
Glosses are generally constructed as follows:

\begin{example}
  \lect Dialect
  \preamble Romanization (undelimited)
  \pronunciation pronunciation
  \gloss
    morphemic & morphemic \\
    transcription & transcription \\
    (object language) & (metalanguage) \\
  \tr Translation
  \source Source
\end{example}

\section{History}\label{sec:history}
\subsection{Internal}\label{sec:hist-int}

\subsection{External}\label{sec:hist-ext}
\langname\ is the spiritual successor to my first conlang, Wei. As with many first attempts at conlanging, I consider Wei to be poorly made, and even more poorly described. Because of this, I consider \langname\ a successor only on account of my initial intentions.

Wei was very anglocentric, only a few steps away from a full relex, if not that. After I pronounced Wei abandoned in September 2019, I took a long break, and used that time to learn more about linguistics. When I returned, I felt at least somewhat more prepared to make something I would be proud of.

When I began work on \langname , I wanted to stray as far away from my original mistakes as I could do comfortably. With that in mind, I outlined just a few goals for myself:

\begin{itemize}
  \item A morphosyntactic alignment differing from the usual Indo-Eu\-ropean nominative--accusative
  \item A minimalistic phonology I would be comfortable pronouncing
  \item Relatively free word order
  \item Some degree of non-concatenative morphology
  \item Thorough documentation, with many examples to bring out nuances in translation
  \item A language that's interesting, fun, and fulfilling to work on
\end{itemize}

I continue to work towards many of those goals as I iterate on the documentation --- including depth of documentation and nuance of lexical entries.

\subsubsection{Typological Details}
While I don't find typology a terribly informative tool in regards to how language \textit{actually} functions, it gives an overview of particularly notable features. As such, here's (another) short bulleted list:

\begin{itemize}
  \item A direct-inverse verbal system (resulting in a syntactically \textit{hierarchical} alignment)
  \item Ergative--absolutive morphology
  \item Strong pro-drop tendencies
  \item (A lot of) reduplication
  \item Sporadic verb serialization
\end{itemize}

Or, in essence, the things which I find interesting.
