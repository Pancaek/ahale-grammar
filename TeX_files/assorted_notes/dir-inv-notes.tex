\section{Direct-Inverse Marking}
Aside from TAM, verbs are inflected as either direct or inverse. This serves in disambiguating syntactic roles of verbal constituents, particularly when they have been partially or completely omitted.

The proper form is selected based off of a person hierarchy. If the agent of a verb is ranked higher on this hierarchy than the patient, the direct form will be used. In the opposite situation, the inverse form is required.

Because of the morphological ergative marking, it is usually straightforward to determine the roles of the constituents, and as a result of this redundancy, many pronouns can be simply dropped and conveyed by this hierarchy.

Generally, the hierarchy is as follows:

{\large 2\textsuperscript{nd} person > 1\textsuperscript{st} person > 3\textsuperscript{rd} person}

 In discourse however, sometimes a first person will be implied where one might otherwise expect second person implication. This does not affect the animacy hierarchy itself, so inversion still works as expected.

In these initial examples, as well as any others which display unexpected behavior based under the general hierarchy, annotations describing agency will be provided and subsequently explained.

%TODO: I haven't decided how I want to typeset those indicators, do that.
\begin{example}
  \preamble meme awale hawi
  \gloss
  me\allo me & ERG\allo1SG \\
  awale & touch \\
  hawi & rabbit \\
  \tr I touch the rabbit.
\end{example}

\begin{example}
  \preamble hahawi awalesi me
  \gloss
  ha\allo hawi & ERG\allo rabbit \\
  awale & touch \\
    -si & INV \\
    me & 1SG \\
    \tr The rabbit touches me.
  \end{example}

  When compound verbs are inflected in the inverse form, the inverse marking is required for each component:

  \begin{example}
    \preamble hahawi awalesi awalesi me
    \gloss
    ha\allo hawi & ERG\allo rabbit \\
    awale & touch \\
    -si & INV \\
    awale & touch \\
    -si & INV \\
    me & 1SG \\
    \tr The rabbit nuzzles me.
\end{example}

Intransitive verbs do not receive the inverse marking; there is no way to compare animacy between things without a pair. Because of this, transitive verbs without an overt 1/2 subject are common, as the subject can be elided and instead conveyed by the verbal marking, with no confusion for an intransitive counterpart.

A similar phonomenon can be found in direct verbs, however different strategies are taken to avoid confusion with ambitransitive verbs.

\begin{example}
  \preamble me kula pa ilau
  \gloss
  me & 1SG \\
  kula & hurt \\
  pa & now \\
  i- & NPST.IPFV \\
  lau & irritate \\
  \tr I got hurt and it's getting worse. %* Remind me to show you meanings of lau cause this is a shit translation atm
  \alt You hurt \underline{me} and it's getting worse.
\end{example}

This example could be interpreted two ways. The intended reading uses \nativetext{ikula} intransitively. However, based on the rules for pronoun elision outlined earlier, the verb could also be interpreted as having an elided second person agent. The position of the object pronoun \nativetext{me} is useful in disambiguating these readings, as typically, (2)>1 forms of \nativetext{kula} would yield \nativetext{kula me}.

If \nativetext{me} is intended as a focused patient, with the elided agent interpreted as such, we would expect the following, with \nativetext{me} explicitly marked as the patient:

\begin{example}
  \preamble lu me kula pa ilau
  \gloss
    lu & OBV \\
    me & 1SG \\
    kula & hurt \\
    pa & now \\
    i- & NPST.IPFV \\
    lau & irritate \\
  \tr You hurt \underline{me} and it's getting worse. %* Remind me to show you meanings of lau cause this is a shit translation atm
\end{example}

\subsection{Irregular Inference}
One place where first person is inferred over second, is when concerning emotions:

\aside{Note that the latter is strange in isolation, it doesn't make sense to declare the emotions of others.}
\begin{examples}
  \baarucols{2}
  \ex
  \preamble nale
  \gloss
  nale & be\_sad \\
  \tr I am sad.
  \ex
  \preamble tu nale
  \gloss
  tu & 2SG \\
  nale & be\_sad \\
  \tr You are sad.
\end{examples}

Or, in expressing wishes, wants and desires:

\begin{example}\label{ex:want-cat}
  \preamble lisu malaku
  \gloss
  lisu & want \\
  malaku & cat \\
  \tr I want a cat.
\end{example}

\aside{\fleuron\ An intransitive reading of \nativetext{lisu} is ungrammatical, so this doesn't introduce any sort of ambiguity in that regard.}
\aside{\fleuron\ Because no subject is overtly present, the indicative assumes the V-inital form usually indicating interrogative mood. These are disambiguated through prosodic changes.}

Alternatively, the object can be fronted and explicitly marked as the patient. This construction restores the visible difference in the interrogative.

\begin{examples}
  \baarucols{2}
  \ex
    \preamble lu malaku lisu
    \gloss
      lu & OBV \\
      malaku & cat \\
      lisu & want \\
    \tr I want a cat.
  \ex
    \preamble lisu malaku
    \gloss
      lisu & want \\
      malaku & cat \\
    \tr Do you want a cat?
\end{examples}

However, in these cases, the interrogative form behaves differently than the indicative reading. Revisiting \baaruref{ex:want-cat}, notice how the indicative form given assumes 1>3 agency, while the interrogative assumes 2>3.
