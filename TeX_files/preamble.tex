\usepackage[outer=6cm, inner=3cm]{geometry}
\usepackage{fontspec}
\usepackage[en-US]{datetime2}

\usepackage{fancyhdr}
\usepackage{parskip}
\usepackage{marginnote}
\usepackage{enumitem}
\usepackage{booktabs}

\usepackage[hidelinks]{hyperref}

%? Ling stuff
\usepackage[nostandards]{baabbrevs}
\usepackage[nostdabbrevs, 2level, columns]{baarux}
\usepackage{tikzvowel, phonrule}
% \baaruset{tr.markup=\transtext, preamble.markup=\bfseries}
% \newbaarulinetype{preamble}{pronunciation}{\phontext}
% \newbaarulinetype{tr}{lit}{\textit{Lit: }}
% \newcommand{\baaruref}[1]{\hyperref[#1]{(\ref*{#1})}}

% \addbaaruhook{everyexesi}{\setlength{\topsep}{0pt}\footnotesize}

% \newbaarulinetype{insert}{tikz}{\tikz[baaru]}
% \tikzset{baaru/.style={baseline=0.3ex}}

%? Date handling
\DTMsavedate{startdate}{2020-03-15}

%? Layout

%* Phonetic rules spacing
% \let\origphonc\phonc
% \def\phonc{\setlength{\tabcolsep}{3pt}\origphonc}
% \let\origphon\phon
% \def\phon{\setlength{\tabcolsep}{3pt}\origphon}

%* Header/footer
\setlength{\headheight}{15pt}
\pagestyle{fancy}
\renewcommand{\chaptermark}[1]{ \markboth{#1}{} }
\renewcommand{\sectionmark}[1]{ \markright{#1} }

\fancyhf{}
\renewcommand{\headrulewidth}{.4pt}
\renewcommand{\footrulewidth}{0pt}
\fancyfoot[LE,RO]{\thepage}
\fancyhead[LE]{\thechapter .\hskip1em\nouppercase{\textsc{\leftmark}}}
\fancyhead[RO]{\thesection .\hskip1em\nouppercase{\rightmark}}

\fancypagestyle{plain}{
  \fancyhf{}
  \renewcommand{\headrulewidth}{0pt}
  \renewcommand{\footrulewidth}{0pt}
}

%* Font
\setmainfont{LibertinusSerif}[
Extension = .otf,
UprightFont = *-Regular,
BoldFont = *-Bold,
ItalicFont = *-Italic,
BoldItalicFont = *-BoldItalic,
]

%* Margin notes
\setlength{\marginparsep}{2em}
\setlength{\marginparpush}{1em}

%* Lists
\setlist{topsep=0em, itemsep=.125em}
\setlist[itemize]{label=---}

%? Shortcuts

%* Ling notation
\newcommand{\langname}{Ahale}
\newcommand{\phontext}[1]{[#1]}
\newcommand{\phomtext}[1]{/#1/}
\newcommand{\morphtext}[1]{|#1|}
\newcommand{\orthotext}[1]{\textlangle #1\textrangle}

\newcommand{\allo}{\textasciitilde}