\part{Syntax}

\chapter{Nominal}

\section{Adpositions}

\begin{description}
  \item[\langword{wu}:] On top (of), from above
  \item[\langword{he}:] Under, from below
  \item[\langword{sa}:] Outside of, from outside
  \item[\langword{i}:] In, from inside
  \item[\langword{ana}:] Next to, near
\end{description}

\ex
\begingl
\glpreamble meme sulau mesikima he line
\pronounced{ˈme.mə ˈsu.lau ˈme.si.ki.ma ˈɣe ˈli.nə}\endpreamble
meme[\textsc{1sg:erg}]
sulau[shoo]
mesikima[\textsc{pl:}fly]
he[under]
line[light]
\glft `I am shooing the flies out from under the light.'
\endgl
\xe

This example, in regards to the preposition itself, is has ambiguous directionality, in that the flies may be shooed either towards or away from the light. Generally this must be inferred, explicit marking of this is uncommon. In the case of \langword{sulau} however, the latter meaning is the only sensible reading. A similar but opposite effect occurs with \langword{kamai} `to send', which is unabiguously read with the former meaning, illustrated below:

\ex
\begingl
\glpreamble mamemalaku ikamai he sesiti
\pronounced{ˈma.mə.ma.la.ku ˈi.ka.mai ˈɣe ˈse.si.ti}\endpreamble
mamemaleku[\textsc{erg:pl:}cat]
ikamai[\textsc{pst.pfv:}send]
he[under]
sesiti[blanket]
\glft `The cats hid under the blanket.'
\endgl
\xe

The verb \langword{kamai} is strictly transitive, so \langword{ta} can be used resumptively to form what is essentially a reflexive construction. Because \langword{ta} does not precede a verb, it also cannot be read as a labile verb.

\ex
\begingl
\glpreamble me ɸene wu ehaixa
\pronounced{ˈme ˈɸe.nə ˈwu ˈe.hai.xa}\endpreamble
me[\textsc{abs:1sg}]
ɸene[live]
wu[on\_top]
ehaixa[\textsc{name}]
\glft `I live in Ehaixa.'
\endgl
\xe

Prepositions are not always translated with their standard definitions. For example, with names of proper nouns, \langword{wu} `on top (of)' is used to express residency. Whereas:

\ex
\begingl
\glpreamble me ɸene i pihane
\pronounced{ˈme ˈɸe.nə ˈi ˈpi.ha.nə}\endpreamble
me[\textsc{abs:1sg}]
ɸene[live]
i[inside]
pihane[house]
\glft `I live in a house.'
\endgl
\xe
This sentence is translated with \langword{i} `inside', using the more typical reading.

A similar dichotomy can be found between \langword{i} and \langword{wu} when in reference to water, particularly its depth. So one might say:

\ex
\begingl
\glpreamble me mulaku i nauwa
\pronounced{ˈme ˈmu.la.ku ˈi ˈnau.wa}\endpreamble
me[\textsc{abs:1sg}]
mulaku[pst.pfv:swin]
i[inside]
nauwa[water]
\glft `I swam underwater.'
\endgl
\xe

This meaning the core argument, \langword{me,} was fully encompassed (ie. submerged at a considerable depth) by the water.

\ex
\begingl
\glpreamble me mulaku wu nauwa
\pronounced{ˈme ˈmu.la.ku ˈwu ˈnau.wa}\endpreamble
me[\textsc{abs:1sg}]
mulaku[pst.pfv:swin]
wu[on\_top]
nauwa[water]
\glft `I swam in the water.'
\endgl
\xe

Whereas here, there is no indication of considerable depth. It should be noted that simply swimming in a deep body of water does not warrant the use of \langword{i,} the depth of water must be relevant to the situation at hand.

\section{Relative Clauses}

\langname\ forms relative clauses with \langword{ʔe}, in this case functioning to associate the dependent clause with the independent one. A sentence like ``I talked to the person that I saw yesterday.'' would be rendered, approximately, as ``I talked to the person \langword{ʔe}, I saw them yesterday.''

\ex
\begingl
\glpreamble meme mahani pane ʔe ta aɸi mauku lu
\pronounced{ˈme.mə ˈma.ha.ni ˈpa.nə ˈʔe ˈta ˈa.ɸi ˈmau.ku ˈlu}\endpreamble
meme[\textsc{1sg:erg}]
mahani[talk\textsc{:pst.pfv}]
pane[person\textsc{:abs}]
ʔe[\textsc{assoc}]+
\nogloss{\lbrack}
ta[prox]
aɸi[yesterday]
mauku[see\textsc{:pst.pfv}]
lu[\textsc{obv}]
\nogloss{\rbrack}
\glft `I talked to the person that I saw yesterday.'
\endgl
\xe

\chapter{Verbal}

\section{Focus}

\ex
\begingl
\glpreamble ɸumau keke hahawi
\pronounced{ˈɸu.mau ˈke.kə ˈɣa.ha.wi}\endpreamble
Ø-ɸumau[\textsc{abs-}grass]
Ø-keke[\textsc{npst.ipfv-}eat]
ha\textasciitilde hawi[\textsc{erg\textasciitilde}rabbit]
\glft `The grass is being eaten by the rabbit.'
\endgl
\xe

Because the grass is still the patient of the verb, it is still marked with the ergative. Fronted arguments of transitive verbs become focused. A passive construction will be used in translation to English. This is solely to approximate the topicalization, as this example is not a true passive (the verb's valency is not decreased). Arguments in default position can be focused, albeit in a different manner. Returning to Example \getfullref{alignment.trns}, but with the agent explicitly focused:

\ex
\begingl
\glpreamble hahawi keke lu ɸumau
\pronounced{ˈɣa.ha.wi ˈke.kə ˈlu ˈɸu.mau}\endpreamble
ha\textasciitilde hawi[\textsc{erg\textasciitilde}rabbit]
keke[\textsc{npst.ipfv-}eat]
lu[\textsc{obv}]
ɸumau[\textsc{abs,}grass]
\glft `The rabbit (as opposed to something else) is eating the grass.'
\endgl
\xe

By marking the already established patient with the obviative\footnotemark, it puts more focus on the (unmarked) proximal argument than would be typical.

\footnotetext{The standard use of proximate/obviate morphology is falling out of use in favor of case marking, Remaining instances have either become fossilized in expressions and idioms, or fulfilled another grammatical purpose, as seen here.}

\section{Labile Verbs}\label{sec:labile_verbs}

A labile verb is a verb that can be either transitive or intransitive, and whose subject when intransitive corresponds to its direct object when transitive. They are also sometimes referred to as ``S=O ambitransitive`` verbs. A prototypical example of this being ``John tripped'' in contrast with ``John tripped Tim''. Unlike a typical ambitransitive verb, the subject's role changes.

\ex
\begingl
\glpreamble ɸihaʔau me
\pronounced{ˈɸi.ha.ʔo ˈme}\endpreamble
ɸihaʔau-∅[trip\textsc{-dir}]
me[\textsc{1sg.abs}]
\glft `You tripped me.'
\endgl
\xe

\ex
\begingl
\glpreamble me ɸiahau
\pronounced{ˈme ˈɸi.a.ho}\endpreamble
me[\textsc{1sg.abs}]
ɸiahau-∅[trip-\textsc{-dir}]
\glft `(You) tripped me.'
\endgl
\xe

These first two examples utilize concepts which have previously been covered in \Sref{sec:person_hierarchy}. The following utilizes the proximate particle, \langword{ta,} in order to mark \langword{me} as the most agentlike argument of a transitive verb. As such, there is no possibility of inferring an agent of \langword{ɸihaʔau.} In this way, labile verbs can be expressed without the need for a dummy agent.

\ex
\begingl
\glpreamble ta me ɸihaʔau
\pronounced{ˈta ˈme ˈɸi.ha.ʔo}\endpreamble
ta[\textsc{prox}]
me[\textsc{1sg.abs}]
ɸihaʔau-∅[trip\textsc{-dir}]
\glft `I tripped.'
\endgl
\xe


\chapter{Discourse}

\section{Discourse Repair}
As an example, lets imagine a hypothetical speaker just said the following:
\ex
\begingl
\glpreamble hahawi keke lu ɸumau
\pronounced{ˈɣa.ha.wi ˈke.kə ˈlu ˈɸu.mau}\endpreamble
ha\textasciitilde hawi[\textsc{erg\textasciitilde}rabbit]
keke[\textsc{npst.ipfv-}eat]
lu[\textsc{obv}]
ɸumau[\textsc{abs,}grass]
\glft `The rabbit (as opposed to something else) is eating the grass'
\endgl
\xe

\subsection{Nominals}

If someone mishears, or for whatever reason needs clarification on the arguments of a transitive verb, the obviative and proximate markers can be used.

\begin{paracol}{2}
If the listener only hears \langword{hahawi ke\-ke lu}, and not the \textit{patient}, the listener can ask the following:
\ex
\begingl
\glpreamble lu?
\pronounced{ˈlu}\endpreamble
lu[\textsc{obv}]
\glft `Eating what?'
\endgl
\xe
\switchcolumn

If a listener only hears \langword{keke lu ɸumau,} and not the \textit{agent}, the listener can ask the following:

\ex
\begingl
\glpreamble ta?
\pronounced{ˈta}\endpreamble
ta[\textsc{prox}]
\glft `What is eating grass?'
\endgl
\xe
\end{paracol}

\subsection{Verbal}
If our listener only heard \langword{``hahawi --- lu ɸumau'',} the response may be:
\ex
\begingl
\glpreamble ta sihu lu?
\pronounced{ˈta ˈsi.hu ˈlu}\endpreamble
ta[\textsc{prox}]
sihu[happen]
lu[\textsc{obv}]
\glft `The rabbit is doing what to grass?'
\endgl
\xe

\langword{ta} and \langword{lu} are used here in a resumptive fashion, rather than repeating the content words. This implies more confidence, in that repeating \langword{hahawi} or \langword{ɸumau} may imply that the listener is also unsure of these components as well, rather than just the verb.

Because this phrase is somewhat of a standard one, it is shortened in colloquial speech. The most aggresive of these shortenings being [ˈtasul(ə)].

\subsection{Responding to Repair Questions}\label{sec:repair_response}

Repair questions can be responded to quite similarly to how a ``standard'' question would be. The main difference is the necessity of the associative particle, \langword{ʔe} to connect \langword{ta} or \langword{lu} to the appropriate content word(s). For example:

\ex
\begingl
\glpreamble ta ʔehawi
\pronounced{ˈta ˈʔe.ha.wi}\endpreamble
ta[\textsc{prox}]
ʔe[\textsc{assoc}]
hawi[rabbit]
\glft `The rabbit (is eating grass)'
\endgl
\xe
\subsection{Self-Correction}
\subsubsection{In conjunction with \langword{ʔe} Ellipsis}

The same structures used to respond to repair questions may be employed to correct or clarify the ellipsed NP, for example if the remaining information  is too ambiguous, incorrect, or is simply no longer relevant.

\ex<adjective_correction>
\begingl
\glpreamble tatakaʔe ala ta ʔekatu kulasi me
\pronounced{ˈta.ta.ka.ʔə ˈa.la ˈta ˈʔe.ka.tu ˈku.la.si ˈme}\endpreamble
tataka[\textsc{erg:}rock]
ʔe[\textsc{assoc}]
ala[white]
ta[\textsc{prox}]
ʔe-katu[ʔe-sharpness]
kulasi[hurt\textsc{:inv}]
me[\textsc{abs:1sg}]
\glft `The big white --- no, sharp --- rock is hurting me'
\endgl
\xe

When \langword{ʔe} is used in discourse repair, it attaches to the correction, rather than to \langword{ta} or \langword{lu.} This the the opposite of when ʔe is used associatively (see \Sref{ch:adjectives}), where \langword{ʔe} attaches to the noun being described.

\subsubsection{New Adjectives}
Similarly, if a noun which previously had no adjectives modifying it needs to have one added in the middle of discourse, the following can be done:

\ex
\begingl
\glpreamble ta nene xase lu ʔe kela
\pronounced{ˈˈta ˈne.nə ˈxa.sə ˈlu.ʔe ˈke.la}\endpreamble
ta[\textsc{prox}]
nene[paint]
xase[spill]
lu[\textsc{obv}]
ʔe[\textsc{assoc}]
kela[green]
\glft `The paint spilled, it's green'
\endgl
\xe

%? Does green exist as its own basic term? Decide.

Special attention should be given when referencing arguments of labile verbs (see \Sref{sec:labile_verbs}). While \langword{ta} marks \langword{nene} as the most agent-like, it is still morphologically and syntactically the verb's patient (note the lack of ergative marking). Because of this, core (S) arguments of labile verbs should be used with the resumptive pronoun used with patients, \langword{lu.}

Also, adjectival additions such as this are strongly preferred to be done \textit{after} the verb, particularly with intransitive verbs. These types of corrections are done differently in that the adjective \textit{is} associated to \langword{lu}. This is different from correcting the use of an incorrect adjective, seen in Example \getfullref{adjective_correction}.