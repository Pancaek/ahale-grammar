
\chapter{Discourse}

\section{Discourse Repair}
If someone mishears, or for whatever reason needs clarification on the arguments of a transitive verb, the obviative and proximate markers can be used.

\ex
\begingl
\glpreamble hahawi keke lu ɸumau
\pronounced{ˈɣa.ha.wi ˈke.kə ˈlu ˈɸu.mau}\endpreamble
ha\textasciitilde hawi[\textsc{erg\textasciitilde}rabbit]
keke[\textsc{npst.ipfv-}eat]
lu[\textsc{obv}]
ɸumau[\textsc{abs,}grass]
\glft `The rabbit (as opposed to something else) is eating the grass'
\endgl
\xe

\begin{paracol}{2}
If the listener only hears \langword{hahawi ke\-ke lu}, and not the \textit{patient}, the listener can ask the following:
\ex
\begingl
\glpreamble lu?
\pronounced{ˈlu}\endpreamble
lu[\textsc{obv}]
\glft `Eating what?'
\endgl
\xe
\switchcolumn

If a listener only hears \langword{keke lu ɸumau,} and not the \textit{agent}, the listener can ask the following:

\ex
\begingl
\glpreamble ta?
\pronounced{ˈta}\endpreamble
ta[\textsc{prox}]
\glft `What is eating grass?'
\endgl
\xe
\end{paracol}

If our listener only heard \langword{``hahawi --- lu ɸumau'',} the response may be:
\ex
\begingl
\glpreamble ta sihu lu?
\pronounced{ˈta ˈsi.hu ˈlu}\endpreamble
ta[\textsc{prox}]
sihu[happen]
lu[\textsc{obv}]
\glft `The rabbit is doing what to grass?'
\endgl
\xe

\langword{ta} and \langword{lu} are used here in a resumptive fashion, rather than repeating the content words. This implies more confidence, in that repeating \langword{hahawi} or \langword{ɸumau} may imply that the listener is also unsure of these components as well, rather than just the verb.

Because this phrase is somewhat of a standard one, it is shortened in colloquial speech. The most aggresive of these shortenings being [ˈtasul(ə)].

\subsection{Responding to Repair Questions}\label{sec:repair_response}

Repair questions can be responded to quite similarly to how a ``standard'' question would be. The main difference is the necessity of the associative particle, \langword{ʔe} to connect \langword{ta} or \langword{lu} to the appropriate content word(s). For example:

\ex
\begingl
\glpreamble ta ʔehawi
\pronounced{ˈta ˈʔe.ha.wi}\endpreamble
ta[\textsc{prox}]
ʔe[\textsc{assoc}]
hawi[rabbit]
\glft `The rabbit (is eating grass)'
\endgl
\xe

The same structures used to respond to repair questions may be employed to correct or clarify the ellipsed NP, for example if the remaining information  is too ambiguous, incorrect, or is simply no longer relevant.

\ex<adjective_correction>
\begingl
\glpreamble tatakaʔe ala ta ʔekatu kulasi me
\pronounced{ˈta.ta.ka.ʔə ˈa.la ˈta ˈʔe.ka.tu ˈku.la.si ˈme}\endpreamble
tataka[\textsc{erg:}rock]
ʔe[\textsc{assoc}]
ala[white]
ta[\textsc{prox}]
ʔe-katu[ʔe-sharpness]
kulasi[hurt\textsc{:inv}]
me[\textsc{abs:1sg}]
\glft `The big white --- no, sharp --- rock is hurting me'
\endgl
\xe

When \langword{ʔe} is used in discourse repair, it attaches to the correction, rather than to \langword{ta} or \langword{lu.} This the the opposite of when ʔe is used associatively (see \Sref{ch:adjectives}), where \langword{ʔe} attaches to the noun being described.

Similarly, if a noun which previously had no adjectives modifying it needs to have one added in the middle of discourse, the following can be done:

\ex
\begingl
\glpreamble ta nene xase lu ʔe kela
\pronounced{ˈˈta ˈne.nə ˈxa.sə ˈlu.ʔe ˈke.la}\endpreamble
ta[\textsc{prox}]
nene[paint]
xase[spill]
lu[\textsc{obv}]
ʔe[\textsc{assoc}]
kela[green]
\glft `The paint spilled, it's green'
\endgl
\xe

Special attention should be given when referencing arguments of labial verbs (see \Sref{sec:labile_verbs}). While \langword{ta} marks \langword{nene} as the most agent-like, it is still morphologically and syntactically the verb's patient (note the lack of ergative marking). Because of this, core (S) arguments of labile verbs should be used with the resumptive pronoun used with patients, \langword{lu.}

Also, adjectival additions such as this are strongly preferred to be done \textit{after} the verb, particularly with intransitive verbs. These types of corrections are done differently in that the adjective \textit{is} associated to \langword{lu}. This is different from correcting the use of an incorrect adjective, seen in Example \getfullref{adjective_correction}.