\chapter{Nominal}

\section{Prepositions}\label{sec:adpositions}

\begin{description}
  \item[\langword{wu}:] On top (of), from above
  \item[\langword{he}:] Under, from below
  \item[\langword{sa}:] Outside of, from outside
  \item[\langword{i}:] In, from inside
  \item[\langword{ana}:] Next to, near
\end{description}

\ex
\begingl
\glpreamble meme sulau mesikima he line
\pronounced{ˈme.mə ˈsu.lau ˈme.si.ki.ma ˈɣe ˈli.nə}\endpreamble
meme[\textsc{1sg:erg}]
sulau[shoo]
mesikima[\textsc{pl:}fly]
he[under]
line[light]
\glft `I am shooing the flies out from under the light.'
\endgl
\xe

This example, in regards to the preposition itself, is has ambiguous directionality, in that the flies may be shooed either towards or away from the light.
Generally this must be inferred, explicit marking of this is uncommon.
In the case of \langword{sulau} however, the latter meaning is the only sensible reading.
A similar but opposite effect occurs with \langword{kamai} `to send', which is unabiguously read with the former meaning, illustrated below:

\ex
\begingl
\glpreamble mamemalaku ikamai (ta) he sesiti
\pronounced{ˈma.mə.ma.la.ku ˈi.ka.mai ˈta ˈɣe ˈse.si.ti}\endpreamble
mamemaleku[\textsc{erg:pl:}cat]
ikamai[\textsc{pst.pfv:}send]
ta[\textsc{prox}]
he[under]
sesiti[blanket]
\glft `The cats hid under the blanket.'\footnotemark
\endgl
\xe

\footnotetext{A more literal translation of this may be `The cats sent themselves under the blanket.', which, while accurately translated, doesn't fit the semantics of the word in either language adequately.}

The verb \langword{kamai} is strictly transitive, so \langword{ta} can be used resumptively to form what is essentially a reflexive construction.
Because \langword{ta} does not precede a verb, it also cannot be read as a labile verb.
As with many uses of resumptive \langword{ta,} it is typically omitted.

\ex
\begingl
\glpreamble meme mukamai mesikima he line
\pronounced{ˈme.mə ˈmu.ka.mai ˈme.si.ki.ma ˈɣe ˈli.nə}\endpreamble
meme[\textsc{1sg:erg}]
mukamai[\textsc{pst.pfv:}send]
mesikima[\textsc{pl:}fly]
he[under]
line[light]
\glft `I shooed the flies into the light.'\footnotemark
\endgl
\xe

When used in this sense, \langword{kamai} implies shooing, just as \langword{sulau} does, but in the opposite direction.
Unlike \langword{sulau,} shooing is not the primary use of \langword{kamai.}

\footnotetext{A more literal translation of this may be `I sent the flies underneath the light.', which is not a very contextually-aware translation.}

When a verb has an ambiguous directionality, and no verb that can fulfill each `direction' (as a pair like \langword{sulau} and \langword{kamai} can in some situations), there are other ways to convey this.
As previously discussed, this directionality is inferred whenever possible, but it can be communicated explicitly if ambiguous, of if clarity is particularly important.

\ex
\begingl
\glpreamble me ɸene wu ehaixa
\pronounced{ˈme ˈɸe.nə ˈwu ˈe.hai.xa}\endpreamble
me[\textsc{abs:1sg}]
ɸene[live]
wu[on\_top]
ehaixa[\textsc{name}]
\glft `I live in Ehaixa.'
\endgl
\xe

Prepositions are not always translated with their standard definitions.
For example, with names of proper nouns, \langword{wu} `on top (of)' is used to express residency.
Whereas:

\ex
\begingl
\glpreamble me ɸene i pihane
\pronounced{ˈme ˈɸe.nə ˈi ˈpi.ha.nə}\endpreamble
me[\textsc{abs:1sg}]
ɸene[live]
i[inside]
pihane[house]
\glft `I live in a house.'
\endgl
\xe
This sentence is translated with \langword{i} `inside', using the more typical reading.

Another dichotomy can be found between \langword{i} and \langword{wu} when in reference to water, particularly its depth.
So one might say:

\ex
\begingl
\glpreamble me mulaku i nauwa
\pronounced{ˈme ˈmu.la.ku ˈi ˈnau.wa}\endpreamble
me[\textsc{abs:1sg}]
mulaku[pst.pfv:swin]
i[inside]
nauwa[water]
\glft `I swam underwater.'
\endgl
\xe

This meaning, the core argument, \langword{me,} was fully encompassed (ie.
submerged at a considerable depth) by the water.

\ex
\begingl
\glpreamble me mulaku wu nauwa
\pronounced{ˈme ˈmu.la.ku ˈwu ˈnau.wa}\endpreamble
me[\textsc{abs:1sg}]
mulaku[pst.pfv:swin]
wu[on\_top]
nauwa[water]
\glft `I swam in the water.'
\endgl
\xe

Whereas here, there is no indication of considerable depth.
It should be noted that simply swimming in a deep body of water does not warrant the use of \langword{i,} the depth of water must be relevant to the situation at hand.

\section{Indirect Objects}
\ex
\begingl
\glpreamble meme i munihasi «keys»
\pronounced{ˈme.mə ˈi ˈmu.ni.ha.si «keys»}\endpreamble
meme[\textsc{1sg:erg}]
i[inside]
munihasi[\textsc{pst.pfv:}give\textsc{:inv}]
«keys»[keys]
\glft `I gave you the keys.'
\endgl
\xe

The verb \langword{niha} `to give' is a ditransitive verb, in which the direct object is the recipient, and the indirect object is the theme.
Both the recipient and theme are marked with the absolutive, but they can be disambiguated in various ways.
Firstly, in a case such as this one, an inspection of the person hierarchy (see \Sref{sec:person_hierarchy}) will suffice.

The verb \langword{niha} has inverse marking, as well as a \textsc{1sg} agent.
Thus, the agent is lower on the person hierarchy than the recipient (direct object), meaning it must be a 2\textsuperscript{nd} person recipient.
Inferring arguments through the person hierarchy assumes the argument is singular, and thus the recipient in this sentence is \textsc{2sg}.
This means that we can be certain that the theme of \langword{niha} is \langword{«keys».}

\ex
\begingl
\glpreamble
\pronounced{}\endpreamble
meme[\textsc{1sg:erg}]
i[inside]
epewu[put]
i[inside]
«bowl»[bowl]
«cup»[cup]
\glft `I am putting the cup in the bowl.'
\endgl
\xe

However, in an example like this one, the person hierarchy is no help in distinguishing the recipient from the theme.
In fact, unless prior context has been established, this construction is completely ambiguous in that regard.
A more expressive construction should be employed in these such cases.

\ex<ditrans_full_marked>
\begingl
\glpreamble
\pronounced{}\endpreamble
meme[\textsc{1sg:erg}]
i[inside]
epewu[put]
i[inside]
lu[\textsc{obv}]
«bowl»[bowl]
«cup»[cup]
\glft `I am putting the cup in the bowl.'
\endgl
\xe

This interaction between locative adpositions and \langword{lu} is noteworthy.
The recipient noun (the direct object) of a ditransitive verb can be marked with \langword{lu} to improve clarity.
However, the thematic recipient (the bowl) is also part of a prepositional phrase.
Generally, \langword{lu} is placed as close to the noun it is describing, particularly when used in this manner.

\subsection{\langword{I} and \langword{Sa} Ditransitives}\label{sec:i_sa_ditrans}
Ditransitive verbs are accompanied by one of two prepositions, \langword{i} or \langword{sa}.
Ordinarily these are used as locatives (see \Sref{sec:adpositions}), though they have special meaning when placed directly before a ditransitive verb.
The construction of ditransitive verb phrases in necessarily much more rigid, in order to accomodate the addition of an indirect object, while still allowing for typical construction of adpositional phrases and related constructions which provide additional description of referents.

Because of the increased rigidity of these constructions, it is usually not possible to focus arguments of ditransitive verbs the same way it can be done with monotransitive verbs (see \Sref{sec:focus} for this).

\subsubsection{\langword{I} Ditransitives}

This is the standard type of ditransitive, and the type that has been used in examples thus far.
\langword{I} ditransitives do not impart any particular focused meaning, and as such are the most common type by far.

\ex
\begingl
\glpreamble meme i munaʔasi memalaku
\pronounced{ˈme.mə ˈi ˈmu.na.ʔa.si ˈmə.ma.la.ku}\endpreamble
meme[\textsc{1sg:erg}]
i[inside]
munaʔasi[\textsc{pst.pfv:}sell\textsc{:inv}]
memalaku[pl:cat]
\glft `I sold you cats.'
\endgl
\xe

\subsubsection{\langword{Sa} Ditransitives}

\ex
\begingl
\glpreamble meme sa munaʔasi memalaku
\pronounced{ˈme.mə ˈsa ˈmu.na.ʔa.si ˈmə.ma.la.ku}\endpreamble
meme[\textsc{1sg:erg}]
sa[outside]
munaʔasi[\textsc{pst.pfv:}sell\textsc{:inv}]
memalaku[pl:cat]
\glft `To you I sold cats.'\footnotemark
\endgl
\xe

\footnotetext{The English translation has only been done this way to approximate the focusing of the recipient.}

When \langword{sa} is used instead of \langword{i} with ditransitive verbs, the focus is shifted to the recipient.
The agent/recipient ordering is very strict, so \langword{sa} is used to convey focus instead.
When \langword{sa} is used this way, it is not accurately named a preposition, and is simply referred to as a particle.
Additional non-morphological focusing information for ditransitive verbs can be found in \Sref{sec:ditrans_focus}.

\section{Relative Clauses}

\langname\ forms relative clauses with \langword{ʔe.} A sentence like ``I talked to the person that I saw yesterday.'' would be rendered, approximately, as ``I talked to the person, I saw them yesterday.''

\ex
\begingl
\glpreamble meme muhanu pane ʔe (ta) aɸi mauku (lu)
\pronounced{ˈme.mə ˈmu.ha.nu ˈpa.nə ˈʔe ˈta ˈa.ɸi ˈmau.ku ˈlu}\endpreamble
meme[\textsc{1sg:erg}]
muhanu[\textsc{pst.pfv:}talk]
pane[\textsc{abs:}person]
ʔe[\textsc{assoc}]+
\nogloss{\lbrack}
ta[prox]
aɸi[yesterday]
mauku[\textsc{pst.pfv:}see]
lu[\textsc{obv}]
\nogloss{\rbrack}
\glft `I talked to the person that I saw yesterday.'
\endgl
\xe

\langword{ʔe} is in this case functioning to associate the dependent clause with the independent one.
Also of note is that here, ʔe is an entirely separate phonetic and prosodic unit, rather than being dependent on the word it attaches to.
This is also acknowledged orthographically to maintain clarity of complex sentences and passages as a whole.

Both \langword{ta} and \langword{lu} are optional here, because there is no other discourse happening with which to confuse the meanings of these two words.
Though, \langword{lu} is marginally less optional in that the transitivity of the verb may sometimes be necessary information to correctly parse the following information.

\chapter{Verbal}

\section{Focus}\label{sec:focus}

\ex
\begingl
\glpreamble ɸumau keke hahawi
\pronounced{ˈɸu.mau ˈke.kə ˈɣa.ha.wi}\endpreamble
Ø-ɸumau[\textsc{abs-}grass]
Ø-keke[\textsc{npst.ipfv-}eat]
ha\textasciitilde hawi[\textsc{erg\textasciitilde}rabbit]
\glft `The grass is being eaten by the rabbit.'
\endgl
\xe

Because the grass is still the patient of the verb, it is still marked with the ergative.
Fronted arguments of transitive verbs become focused.
A passive construction will be used in translation to English.
This is solely to approximate the topicalization, as this example is not a true passive (the verb's valency is not decreased).
Arguments in default position can be focused, albeit in a different manner.
Returning to Example \getfullref{alignment.trns}, but with the agent explicitly focused:

\ex
\begingl
\glpreamble hahawi keke lu ɸumau
\pronounced{ˈɣa.ha.wi ˈke.kə ˈlu ˈɸu.mau}\endpreamble
ha\textasciitilde hawi[\textsc{erg\textasciitilde}rabbit]
keke[\textsc{npst.ipfv-}eat]
lu[\textsc{obv}]
ɸumau[\textsc{abs,}grass]
\glft `The rabbit (as opposed to something else) is eating the grass.'
\endgl
\xe

By marking the already established patient with the obviative\footnotemark, it puts more focus on the (unmarked) proximal argument than would be typical.

\footnotetext{The standard use of proximate/obviate morphology is falling out of use in favor of case marking, Remaining instances have either become fossilized in expressions and idioms, or fulfilled another grammatical purpose, as seen here.}

\subsection{Focusing of Ditransitive Verb Arguments}\label{sec:ditrans_focus}

Aside from the morpho­logical strategies for manipulating focus found in \Sref{sec:i_sa_ditrans}, a small amount of focusing can still be done through syntactic means.
Primarily, focusing the theme of a ditransitive verb (which in the case of \langname\ is analyzed as it's indirect object).

To illustrate this, we can look back at Example \getfullref{ditrans_full_marked}, reporduced below:

\ex
\begingl
\glpreamble
\pronounced{}\endpreamble
meme[\textsc{1sg:erg}]
i[inside]
epewu[put]
i[inside]
lu[\textsc{obv}]
«bowl»[bowl]
«cup»[cup]
\glft `I am putting the cup in the bowl.'
\endgl
\xe

Because the role of every argument is clear in this sentence, the theme can simply be fronted to focus it, as with the standard means of shifting focus:

\ex
\begingl
\glpreamble
\pronounced{}\endpreamble
«cup»[cup]
meme[\textsc{1sg:erg}]
i[inside]
epewu[put]
i[inside]
lu[\textsc{obv}]
«bowl»[bowl]
\glft `I am putting the \textit{cup} in the bowl.'\footnotemark
\endgl
\xe

\footnotetext{There is not a straightforward way to express this focus in a distinct English sentence, as has been done previously.
  As such, italics have been employed to mark the focused element.}

\section{Labile Verbs}\label{sec:labile_verbs}

A labile verb is a verb that can be either transitive or intransitive, and whose subject when intransitive corresponds to its direct object when transitive.
They are also sometimes referred to as ``S=O ambitransitive`` verbs.
A prototypical example of this being ``John tripped'' in contrast with ``John tripped Tim''.
Unlike a typical ambitransitive verb, the subject's role changes.

\ex
\begingl
\glpreamble ɸihaʔau me
\pronounced{ˈɸi.ha.ʔo ˈme}\endpreamble
ɸihaʔau-∅[trip\textsc{-dir}]
me[\textsc{1sg.abs}]
\glft `You tripped me.'
\endgl
\xe

\ex
\begingl
\glpreamble me ɸiahau
\pronounced{ˈme ˈɸi.a.ho}\endpreamble
me[\textsc{1sg.abs}]
ɸiahau-∅[trip-\textsc{-dir}]
\glft `(You) tripped me.'
\endgl
\xe

These first two examples utilize concepts which have previously been covered in \Sref{sec:person_hierarchy}.
The following utilizes the proximate particle, \langword{ta,} in order to mark \langword{me} as the most agentlike argument of a transitive verb.
As such, there is no possibility of inferring an agent of \langword{ɸihaʔau.} In this way, labile verbs can be expressed without the need for a dummy agent.

\ex
\begingl
\glpreamble ta me ɸihaʔau
\pronounced{ˈta ˈme ˈɸi.ha.ʔo}\endpreamble
ta[\textsc{prox}]
me[\textsc{1sg.abs}]
ɸihaʔau-∅[trip\textsc{-dir}]
\glft `I tripped.'
\endgl
\xe
