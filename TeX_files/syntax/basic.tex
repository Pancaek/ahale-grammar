\chapter{Nominal}

\section{Adpositions}

\begin{description}
  \item[\langword{wu}:] On top (of), from above
  \item[\langword{he}:] Under, from below
  \item[\langword{sa}:] Outside of, from outside
  \item[\langword{i}:] In, from inside
  \item[\langword{ana}:] Next to, near
\end{description}

\ex
\begingl
\glpreamble meme sulau mesikima he line
\pronounced{ˈme.mə ˈsu.lau ˈme.si.ki.ma ˈɣe ˈli.nə}\endpreamble
meme[\textsc{1sg:erg}]
sulau[shoo]
mesikima[\textsc{pl:}fly]
he[under]
line[light]
\glft `I am shooing the flies out from under the light.'
\endgl
\xe

This example, in regards to the preposition itself, is has ambiguous directionality, in that the flies may be shooed either towards or away from the light. Generally this must be inferred, explicit marking of this is uncommon. In the case of \langword{sulau} however, the latter meaning is the only sensible reading. A similar but opposite effect occurs with \langword{kamai} `to send', which is unabiguously read with the former meaning, illustrated below:

\ex
\begingl
\glpreamble mamemalaku ikamai he sesiti
\pronounced{ˈma.mə.ma.la.ku ˈi.ka.mai ˈɣe ˈse.si.ti}\endpreamble
mamemaleku[\textsc{erg:pl:}cat]
ikamai[\textsc{pst.pfv:}send]
he[under]
sesiti[blanket]
\glft `The cats hid under the blanket.'
\endgl
\xe

The verb \langword{kamai} is strictly transitive, so \langword{ta} can be used resumptively to form what is essentially a reflexive construction. Because \langword{ta} does not precede a verb, it also cannot be read as a labile verb.

\ex
\begingl
\glpreamble me ɸene wu ehaixa
\pronounced{ˈme ˈɸe.nə ˈwu ˈe.hai.xa}\endpreamble
me[\textsc{abs:1sg}]
ɸene[live]
wu[on\_top]
ehaixa[\textsc{name}]
\glft `I live in Ehaixa.'
\endgl
\xe

Prepositions are not always translated with their standard definitions. For example, with names of proper nouns, \langword{wu} `on top (of)' is used to express residency. Whereas:

\ex
\begingl
\glpreamble me ɸene i pihane
\pronounced{ˈme ˈɸe.nə ˈi ˈpi.ha.nə}\endpreamble
me[\textsc{abs:1sg}]
ɸene[live]
i[inside]
pihane[house]
\glft `I live in a house.'
\endgl
\xe
This sentence is translated with \langword{i} `inside', using the more typical reading.

A similar dichotomy can be found between \langword{i} and \langword{wu} when in reference to water, particularly its depth. So one might say:

\ex
\begingl
\glpreamble me mulaku i nauwa
\pronounced{ˈme ˈmu.la.ku ˈi ˈnau.wa}\endpreamble
me[\textsc{abs:1sg}]
mulaku[pst.pfv:swin]
i[inside]
nauwa[water]
\glft `I swam underwater.'
\endgl
\xe

This meaning the core argument, \langword{me,} was fully encompassed (ie. submerged at a considerable depth) by the water.

\ex
\begingl
\glpreamble me mulaku wu nauwa
\pronounced{ˈme ˈmu.la.ku ˈwu ˈnau.wa}\endpreamble
me[\textsc{abs:1sg}]
mulaku[pst.pfv:swin]
wu[on\_top]
nauwa[water]
\glft `I swam in the water.'
\endgl
\xe

Whereas here, there is no indication of considerable depth. It should be noted that simply swimming in a deep body of water does not warrant the use of \langword{i,} the depth of water must be relevant to the situation at hand.

\section{Relative Clauses}

\langname\ forms relative clauses with \langword{ʔe.} A sentence like ``I talked to the person that I saw yesterday.'' would be rendered, approximately, as ``I talked to the person, I saw them yesterday.''

\ex
\begingl
\glpreamble meme mahani pane ʔe (ta) aɸi mauku (lu)
\pronounced{ˈme.mə ˈma.ha.ni ˈpa.nə ˈʔe ˈta ˈa.ɸi ˈmau.ku ˈlu}\endpreamble
meme[\textsc{1sg:erg}]
mahani[talk\textsc{:pst.pfv}]
pane[person\textsc{:abs}]
ʔe[\textsc{assoc}]+
\nogloss{\lbrack}
ta[prox]
aɸi[yesterday]
mauku[see\textsc{:pst.pfv}]
lu[\textsc{obv}]
\nogloss{\rbrack}
\glft `I talked to the person that I saw yesterday.'
\endgl
\xe

\langword{ʔe} is in this case functioning to associate the dependent clause with the independent one. Also of note is that here, ʔe is an entirely separate phonetic and prosodic unit, rather than being dependent on the word it attaches to. This is also acknowledged orthographically to maintain clarity of complex sentences and passages as a whole.

Both \langword{ta} and \langword{lu} are optional here, because there is no other discourse happening with which to confuse the meanings of these two words. Though, \langword{lu} is marginally less optional in that the transitivity of the verb may sometimes be necessary information to correctly parse the following information.


\chapter{Verbal}

\section{Focus}

\ex
\begingl
\glpreamble ɸumau keke hahawi
\pronounced{ˈɸu.mau ˈke.kə ˈɣa.ha.wi}\endpreamble
Ø-ɸumau[\textsc{abs-}grass]
Ø-keke[\textsc{npst.ipfv-}eat]
ha\textasciitilde hawi[\textsc{erg\textasciitilde}rabbit]
\glft `The grass is being eaten by the rabbit.'
\endgl
\xe

Because the grass is still the patient of the verb, it is still marked with the ergative. Fronted arguments of transitive verbs become focused. A passive construction will be used in translation to English. This is solely to approximate the topicalization, as this example is not a true passive (the verb's valency is not decreased). Arguments in default position can be focused, albeit in a different manner. Returning to Example \getfullref{alignment.trns}, but with the agent explicitly focused:

\ex
\begingl
\glpreamble hahawi keke lu ɸumau
\pronounced{ˈɣa.ha.wi ˈke.kə ˈlu ˈɸu.mau}\endpreamble
ha\textasciitilde hawi[\textsc{erg\textasciitilde}rabbit]
keke[\textsc{npst.ipfv-}eat]
lu[\textsc{obv}]
ɸumau[\textsc{abs,}grass]
\glft `The rabbit (as opposed to something else) is eating the grass.'
\endgl
\xe

By marking the already established patient with the obviative\footnotemark, it puts more focus on the (unmarked) proximal argument than would be typical.

\footnotetext{The standard use of proximate/obviate morphology is falling out of use in favor of case marking, Remaining instances have either become fossilized in expressions and idioms, or fulfilled another grammatical purpose, as seen here.}

\section{Labile Verbs}\label{sec:labile_verbs}

A labile verb is a verb that can be either transitive or intransitive, and whose subject when intransitive corresponds to its direct object when transitive. They are also sometimes referred to as ``S=O ambitransitive`` verbs. A prototypical example of this being ``John tripped'' in contrast with ``John tripped Tim''. Unlike a typical ambitransitive verb, the subject's role changes.

\ex
\begingl
\glpreamble ɸihaʔau me
\pronounced{ˈɸi.ha.ʔo ˈme}\endpreamble
ɸihaʔau-∅[trip\textsc{-dir}]
me[\textsc{1sg.abs}]
\glft `You tripped me.'
\endgl
\xe

\ex
\begingl
\glpreamble me ɸiahau
\pronounced{ˈme ˈɸi.a.ho}\endpreamble
me[\textsc{1sg.abs}]
ɸiahau-∅[trip-\textsc{-dir}]
\glft `(You) tripped me.'
\endgl
\xe

These first two examples utilize concepts which have previously been covered in \Sref{sec:person_hierarchy}. The following utilizes the proximate particle, \langword{ta,} in order to mark \langword{me} as the most agentlike argument of a transitive verb. As such, there is no possibility of inferring an agent of \langword{ɸihaʔau.} In this way, labile verbs can be expressed without the need for a dummy agent.

\ex
\begingl
\glpreamble ta me ɸihaʔau
\pronounced{ˈta ˈme ˈɸi.ha.ʔo}\endpreamble
ta[\textsc{prox}]
me[\textsc{1sg.abs}]
ɸihaʔau-∅[trip\textsc{-dir}]
\glft `I tripped.'
\endgl
\xe
