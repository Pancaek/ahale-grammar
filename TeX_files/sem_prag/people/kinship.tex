\section{Kinship}

\subsection{Classification}
\langname\ can be broadly categorized as having a Hawaiian kinship system, wherein cousins share the same terms as siblings, and aunt/uncle do not have distinct terminology from parents. In a traditional Hawaiian kinship system, gender is distinguished, but this is not the case in \langname .

Age is used as the secondary division of kinship terminology, wherein younger sibling/older sibling are each a distinct lexeme, rather than being formed through adjectives or other derivational means. The same can be said of parent/aunt/uncle, which share one term, aside from division by age.

The following table summarizes a few of the most basic terms used within \langname 's kinship system. Note that whenever possible, native terms will be used to avoid clumsy explanation.

\begin{table}[ht]
  \centering
  \begin{tabular}{lll}
    \toprule
                      & Same/Younger      & Older           \\ \midrule
    Child             & \langword{ausuʔi} & \langword{teme} \\
    Sibling           & \langword{naʔuwe} & \langword{wase} \\
    Parent/Aunt/Uncle & \langword{kausu}  & \langword{mele} \\ \bottomrule
  \end{tabular}
  \caption{Basic Kinship Terminology}
\end{table}

\subsection{Age Distinction}
\langname 's age distinction is based on relative age of other members within the generation. That is to say, if one's mother is younger than one's father, `young-parent' will be used for her, and vice-versa.

If the people involved were an aunt (who is younger than the respective uncle), and the mother (also younger than the respective father) of the ego,\footnotemark\ things become slightly ambiguous, without further explanation: Are both referred to as `young-parent', or are their ages compared as well?

The former is indeed the case. Consider the following examples in relation to an ego with two \langword{naʔuwe} and one \langword{wase}:

\ex<kin_inv>
\begingl
naʔuwe[young\_sibling]
kulasi[hurt\textsc{-inv}]
naʔuwe[young\_sibling]
\glft `My (younger) sibling is hitting their (older) sibling.'
\endgl
\xe

\footnotetext{The ego is the person from which kinship terms are arranged relative to.}

This raises a few points of interest:
\begin{itemize}
  \item Unless specified within a conversation, kinship terms are always used with the expectation that the ego is the \textit{speaker,} rather than the subject
  \item Verbal morphology can be indicative of further granularity which is not evident solely from the terms themselves
\end{itemize}

The first point is relatively evident, made so by the use of \langword{naʔuwe} for both referents, coupled with the translation. This tells us a lot, in fact, based upon the premise established above.

Firstly, we can deduce that the \langword{naʔuwe} which is the subject is the youngest of the ego and between the \langword{menaʔuwe} and \langword{me.} This is because of the inverse marking, which allows the observation that the first \langword{naʔuwe} is younger than the second. The use of \langword{naʔuwe} itself informs us of the relationship to the ego in such a situation.

Also, unlike most other nouns, kinship terms are assumed to be implicitly related to each other, hence the above does not refer to a \langword{naʔuwe} who hits a \langword{naʔuwe} from another family.

\subsection{Utterance Parroting with \langword{Ihesu}}
The strategy used above is fragile, since the speaker must reframe situations in which they are not the ego, and this is not always desirable. A special verb, \langword{ihesu,} can be used to quote, or `parrot' the speech of another person, in which case the ego shifts to the quoted person rather than the speaker as normally would be the case. This should not be used in all situations, as \langword{hanu} is often enough when simply repeating a phrase, whereas \langword{ihesu} serves to shift perspective completely.

\ex<parroting>
\begingl
tutu[\textsc{erg\textasciitilde 2sg}]
mukula[\textsc{pst.pfv-}hurt]
ihesu[ihesu]
lu[lu]
ausuʔi[young\_child]
kausu[young\_parent]
\glft `\,``You hurt my child!'', is what my parent said.'
\endgl
\xe

In situations where both \langword{ihesu} and \langword{hanu} could be used, \langword{ihesu} is preferred for formal purposes, such as when a courier delivers a message, speaking for the person having sent it, voicing or `parroting'--- as I've used here, the message for the original sender. Whereas \langword{hanu} still requires the original speaker as the agent, and has slightly more reportative connotations.

\subsection{Morphosyntactic Implications}
Kinship terms, and occassionally the names of specific people, can change how a sentence must be constructed. At the most basic level, these terms and their relationships within a phrase are of direct consequence to the alignment of the verbs they are used in. Older generations are seen as higher on the person hierarchy (see \Sref{sec:person_hierarchy} for general guidelines about this), and therefore will have affect on verbal morphology.

It should be noted though that this is not applicable when the referents are of different families. That is to say, a grandparent of hypothetical family A will never outrank a parent from hypothetical family B (or any other family member). These considerations should only be made from within a single family.

From within a generation, all members are on the same level of the person hierarchy (except for exceptions as with Example \getfullref{kin_inv}), and as such will use the form of the verb as would be done ordinarily.
