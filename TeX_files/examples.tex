\part{Example Sentences}
% \ex
% \begingl
% masa[sun]
% «shine»[\textsc{npst.ipfv}-shine]
% \glft `The sun shines.'
% \endgl
% \xe

% \ex
% \begingl
% masa[sun]
% pa[now]
% «shine»[\textsc{npst.ipfv}-shine]
% \glft `The sun is shining.'
% \endgl
% \xe

% \ex
% \begingl
% masa[sun]
% i-«shine»[\textsc{pst.ipfv}-shine]
% \glft `The sun shone.'
% \endgl
% \xe

% \pex
% \a
% \begingl
% masa[sun]
% «shine»[\textsc{npst.ipfv}-shine]
% \glft `The sun will shine.'
% \endgl
% \a
% \begingl
% masa[sun]
% i-ti[\textsc{pst.ipfv-}shine,]
% alete[thus]
% ∅-ti[\textsc{npst.ipfv-}shine]
% \glft `The sun will shine.'
% \endgl
% \xe

% \pex
% \a
% \beginglpanel
% \glpreamble
% \endpreamble
% masa[sun]
% pa[now]
% «shine»[\textsc{npst.ipfv}-shine]
% \endgl
% This usage of the imperfective typically requires a true duration, and as such this construction as written may be confused for a standard progressive reading, as seen above.
% \endpanel
% `The sun has been shining.'

% \a
% \beginglpanel
% \glpreamble
% \endpreamble
% masa[sun]
% neʔe[hour]
% «shine»[\textsc{npst.ipfv}-shine]
% \endgl
% This usage of the imperfective will be much more commonly encountered.
% \endpanel
% `The sun has been shining for one hour.'
% \xe

% \ex
% \begingl
% masa[sun]
% «tomorrow»[tomorrow]
% «shine»[\textsc{npst.ipfv}-shine]
% \glft `The sun will shine tomorrow.'
% \endgl
% \xe

% % The sun shines brightly.

\ex
\begingl
\glpreamble kula! aʔahisiʔe tetaʔa, ta lixi nasu ɸa me\endpreamble
kula[\textsc{intj}]
aʔahisi[\textsc{erg:}wind]
ʔe[\textsc{assoc}]
tetaʔa[cold]
ta[prox]
lixi[freeze]
nasu[nose]
ɸa[\textsc{poss}]
me[\textsc{1sg}]
\glft `Whew! that cold wind freezes my nose!'
\endgl
\xe
