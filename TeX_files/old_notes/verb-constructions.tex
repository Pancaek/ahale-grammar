
\chapter{Verb Constructions}

\section{Reciprocation}
\begin{example}
  \preamble meme alusi tu, lu siha ta
  %! There's a bug causing this arrow to not point correctly
  % \tikz { \draw[->] (0,0) (\PtA,0)--(\PtA,1ex)-|(\PtB,0) ; }
  \gloss
    me\allo me#B & ERG\allo 1SG \\
    alu & talk \\
    -si & INV \\
    lu & OBV \\
    siha & happen \\
    ta#A & PROX \\
  \tr I talk to you, you talk to me.
\end{example}

Reciprocation makes heavy use of the anaphoric pronouns \nativetext{ta} and \nativetext{lu} as to avoid explicit duplication of the constituents.

This is in part due to the fact that this particular construction when expressed without anaphora is considered to encode multiple events, rather than one singular reciprocated event.

\nativetext{Siha} assumes the role of a dummy verb, such that the original verb should not be repeated.

In the most innovative of dialects, \nativetext{siha} may occur without any additional arguments, such that \nativetext{siha} following a verb phrase comes to denote a reciprocal.

\begin{example}
  \preamble alusi siha
  \gloss
  alu & talk \\
  -si & INV \\
  siha & happen \\
  \tr We're chatting.
\end{example}

However, as a result of its origins, the inverse form is strongly preferred in conveying 1/2 reciprocation, \notabletext{even in cases which would ordinarily use direct verbs on account of 2>1 animacy}.

Other reciprocals require the complete \nativetext{lu siha ta} construction:

\begin{example}
  \preamble memele ihesu pane xa lu teme lu siha ta
  %! There's a bug causing this arrow to not point correctly
  % \tikz { \draw[->] (0,0) (\PtA,0)--(\PtA,1ex)-|(\PtB,0) ; }
  \gloss
    me\allo mele & ERG\allo older\_parent \\
    ihesu & ihesu \\
    pane & person \\
    xa & love \\
    lu & OBV \\
    teme#A & older\_child \\
    lu#B & OBV \\
    siha & happen \\
    ta & PROX \\
  \tr The parent and their child love each other.
\end{example}
