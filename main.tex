\documentclass[openany, 12pt, b5paper, draft]{memoir}
\usepackage{fontspec}
\let\ordinal\relax %! Fixes \ordinal conflict with memoir
\usepackage[us]{datetime}
\newdate{startdate}{15}{3}{2020}
\usepackage{paracol}

\usepackage{expex, tikzvowel, phonrule}


\usetikzlibrary{fit}
\newdimen\contourraise
\newdimen\contourspacetokenwidth
\newcount\lasttokennumber
\newcount\currenttokennumber
\newcount\contourmarkcount
\newcount\contourtokenunderlinestate
\newbox\contourbox

\tikzset{
  tight fit/.style={
      inner sep=0pt,
      outer sep=0pt,
    },
  %
  %
  % How far above the reference anchor of the text,
  contour raise/.code=\pgfmathsetlength\contourraise{#1},
  contour reference anchor/.store in=\contourreferenceanchor,
  contour reference anchor=base east,
  % The `scale' for the values in the contour height specification
  contour scale/.store in=\contourscale,
  contour scale=3pt,
  % The prefix for the contour marks.
  contour mark prefix/.store in=\contourmarkprefix,
  contour mark prefix=contour,
  % The style for the contour path
  contour/.style={
      draw,
      rounded corners=1ex,
    },
  % The style for the token nodes
  every contour token/.style={
      anchor=base west,
      inner sep=0pt,
    },
  contour underline/.style={
      draw
    },
  % The character to insert a mark (use with care)
  contour mark character/.store in=\contourmarkchar,
  contour mark character=|,
  % Want to change the code for contour marks? Use this key.
  contour mark code/.store in=\contourmarkcode,
  % Want to change the code for tokens? Use this key.
  contour token code/.store in=\contourtokencode,
  % Want to change the code for drawing the contour? Use this  key.
  contour code/.store in=\contourcode,
  %
  % Default stuff
  contour mark code={%
      \coordinate (\contourmarkprefix-\the\contourmarkcount)
      at ([yshift=\contourraise, y=\contourscale,
        shift={(0,\currentcontourheight)}]token-\the\currenttokennumber.\contourreferenceanchor);
    },
  contour token code={%
      \node [every contour token/.try] at
      (token-\the\lasttokennumber.base east)
      (token-\the\currenttokennumber) {\token};
    },
  contour code={
      \draw [contour] (\contourmarkprefix-1)
      \foreach \y in {2,...,\the\contourmarkcount}{ --
          (\contourmarkprefix-\y) };
    },
  %
  % Don't draw the contour.
  tokens only/.style={
      contour code={}
    },
  %
  % Only draw the contour (but the space is still used for the tokens)
  contour only/.style={
      every contour token/.append style={
          execute at begin node={\setbox\contourbox=\hbox\bgroup},
          execute at end node=\egroup\phantom{\box\contourbox}%
        },
      underline/.style={
          draw=none
        }
    },
  %
  % Make tokens follow the contour marks.
  tokens follow contour/.style={
      tokens only,
      contour token code={%
          \node [every contour token/.try, y=\contourscale] at
          (token-\the\lasttokennumber.base east |-
          0,\currentcontourheight)
          (token-\the\currenttokennumber) {\token};
        },
    },
  % What style to use when drawing underline
  underline/.style={
      draw
    },
  % The underline is drawn along the south side of a node which
  % takes this style.
  underline token/.style={
      inner ysep=1pt
    },
  % When grouping tokens (e.g., for putting box around)
  % this style is applied to a node that is fitted around the group
  token group/.style={
      inner xsep=1pt,
      inner ysep=2pt,
      rounded corners=2pt
    },
  % Draw boxes around tokens groups.
  box tokens/.style={
      token group/.append style={
          draw
        }
    },
  % Change the width of the spaces.
  space token width/.code=\pgfmathsetlength\contourspacetokenwidth{#1},
  space token width=0.125cm
}

\makeatletter

\def\at@{@}

\newcommand\contour[2][]{%
  \begin{scope}[#1]
    \coordinate (token-0);
    \currenttokennumber=0\relax%
    \lasttokennumber=0\relax%
    \contourmarkcount=0\relax%
    \def\lastcontourheight{0}%
    \contourtokenunderlinestate=0\relax%
    \@contour#2@%
    }

    % Must check for a spaces
    \def\@contour{\futurelet\@token\@checkforspace}

    \def\@uscore{_}
    \def\@checkforspace{%
      \ifx\@token\@sptoken%
        \let\@next=\@replacespace%
      \else%
        \if\@token\contourmarkchar%
          \let\@next=\@contour@insertmark
        \else%
          \if\@token\@uscore
            \let\@next=\@contourtoggleunderline%
          \else%
            \let\@next=\@@contour%
          \fi%
        \fi%
      \fi%
      \@next%
    }

    \def\@contourtoggleunderline#1{%
      \advance\contourtokenunderlinestate by1\relax
      \ifnum\contourtokenunderlinestate>3\relax%
        \contourtokenunderlinestate=0\relax%
      \fi%
      \@contour%
    }

    \def\@contour@insertmark{%
      \afterassignment\@@contour@insertmark\let\@token=%
    }

    \def\@@contour@insertmark{%
      \futurelet\@token\@@@contour@insertmark}%

    \def\@@@contour@insertmark{%
    \if\@token[%
      \let\@next=\@@@@contour@insertmark%
    \else%
      \let\currentcontourheight=\lastcontourheight%
      \let\@next=\@@@@@contour@insertmark%
    \fi%
    \@next%
    }

    \def\@@@@contour@insertmark[#1]{%
      \def\@tmp{#1}%
      \ifx\@tmp\@empty%
        \let\currentcontourheight=\lastcontourheight%
      \else%
        \def\currentcontourheight{#1}%
      \fi%
      \@@@@@contour@insertmark}

    \def\@@@@@contour@insertmark{%
      \advance\contourmarkcount by1\relax%
      % Code for inserting mark
      \contourmarkcode%
      \let\lastcontourheight=\currentcontourheight%
      \@contour}

    \def\contourspacetoken{{\hbox to \contourspacetokenwidth{\hfill}}}

    \def\@replacespace#1{%
      \@contour\contourspacetoken#1%
    }

    \def\@@contour#1{%
      \def\@token{#1}%
      \if\@token\at@%
        \@contourdounderline%
        \pgfutil@ifundefined{pgf@sh@ns@tokengroup}{}{%
          \node [tight fit, fit={(tokengroup)}, token group/.try] {};
          \global\let\pgf@sh@ns@tokengroup=\relax%
        }%
        \let\@next=\@@@contour%
      \else%
        \lasttokennumber=\currenttokennumber%
        \advance\currenttokennumber by1%
        \let\token=\@token%
        % Code for typesetting token
        \contourtokencode%
        % Manage underline state
        \@contourdounderline%
        \def\@@token{\contourspacetoken}%
        \ifx\@token\@@token%
          \pgfutil@ifundefined{pgf@sh@ns@tokengroup}{}{%
            \pgfutil@ifundefined{pgf@sh@ns@underline}{}{%
              \node [tight fit, fit={(tokengroup) (underline)}]
              (tokengroup)
              {};}%
            \node [tight fit, fit={(tokengroup)}, token group/.try] {};
            \global\let\pgf@sh@ns@tokengroup=\relax%
          }%
        \else
          \pgfutil@ifundefined{pgf@sh@ns@tokengroup}{%
            \node [tight fit,
              fit={(token-\the\currenttokennumber)}]
            (tokengroup) {};
          }{%
            \node [tight fit,
              fit={(token-\the\currenttokennumber)
                  (tokengroup)}]
            (tokengroup){};
          }%
        \fi%
        \let\@next=\@contour
        %
      \fi%
      \@next%
    }

    \def\@contourdounderline{%
      \ifcase\contourtokenunderlinestate%
      \or
        \node [tight fit, fit={(token-\the\currenttokennumber)}]
        (underline) {};
        \contourtokenunderlinestate=2\relax%
      \or%
        \node [tight fit,fit={(token-\the\currenttokennumber) (underline)}]
        (underline) {};
      \or%
        \node [tight fit, fit={(underline)}, underline token/.try]
        (underline) {};
        \draw [underline/.try]
        (underline.south west) -- (underline.south east);
        \pgfutil@ifundefined{pgf@sh@ns@tokengroup}{}{%
          \node [tight fit, fit={(tokengroup) (underline)}]
          (tokengroup) {};%
          \node [tight fit, fit={(tokengroup)}, token group/.try] {};
          \global\let\pgf@sh@ns@tokengroup=\relax%
          \global\let\pgf@sh@ns@underline=\relax%
        }
        \contourtokenunderlinestate=0\relax
      \fi%
    }
    \def\@@@contour{%
    \ifnum\contourmarkcount>1
      % Code for drawing contour
      \contourcode%
    \fi%
  \end{scope}%
}

\makeatother


\usepackage[hidelinks]{hyperref}
\hypersetup{linktoc=all}

\titlingpageend{\clearforchapter}{\clearforchapter} %! Titlingpage/openany fix for blank page
\pagestyle{ruled}
\makeoddfoot{plain}{}{}{\thepage}
\makeevenfoot{plain}{\thepage}{}{}
\copypagestyle{title}{empty}

\setmainfont{Libertinus Serif}

\setlength{\parindent}{0pt}
\nonzeroparskip

\lingset{
  glstyle=nlevel,
  belowglpreambleskip={-.5\parskip},
  aboveglftskip={-.5\parskip},
  exskip={0pt},
  glneveryline={,\it,\nogloss}
}

% Doc specific stuff
\newcommand{\langword}[1]{\textit{#1}}
\newcommand{\langname}{Ahale}

\newcommand{\pronounced}[1]{\vskip-0.5\parskip[#1]}

\begin{document}
\title{\langname : A Complete Reference}
\author{Pancake}
\date{\displaydate{startdate} - \today}
\frontmatter
\begin{titlingpage}
  \maketitle
\end{titlingpage}

\begin{KeepFromToc}
  \tableofcontents
\end{KeepFromToc}
\mainmatter

\part{Phonology}\label{prt:phonology}
\chapter{Segmental Phonology}\label{ch:seg-phono}
\section{Inventory}\label{sec:phono-inv}

\begin{table}[ht]
  \centering
  \begin{tabular}{*{7}{c}}
    \toprule
    & Labial & Alveolar & Velar & Glottal \\\midrule
    Plosive   & p      & t        & k     & ʔ       \\
    Nasal     & m      & n        &       &         \\
    Fricative & ɸ      & s        & x     & h       \\
    Sonorant  & w      & l        &       &         \\
    \bottomrule
  \end{tabular}
  \caption{Consonant Inventory}
  \label{table:consonants}
\end{table}

\aside{\fleuron \phontext{e} is an allophone of \phomtext{ə}, rather than a phoneme unto itself as it may first appear, (see \textsc{rel} and \textsc{assoc} \nativetext{ʔe})
}

\begin{figure}[ht]
  \centering
  \begin{vowel}
    \vpoint{0}{0}{i}
    \vpoint{0}{2}{u}
    \vpoint{1.5}{1}{ə}
    \vpoint{3}{1}{a}
    \varrow{a}{u}
    \varrow{a}{i}
  \end{vowel}
  \caption{Phonemic Vowel Inventory}
  \label{table:vowel_phonemes}
\end{figure}

Diphthongs consisting of \phomtext{ə} + high vowel can be observed in a few words, however these diphthongs are prone to collapse. In all but the most conservative of dialects, these are realized as \phomtext{əi əu} \phontext{e o}.

\vfill %! This is not smart don't do this
\section{Stress}
\subsection{Assignment}
Stress is generally assigned to the initial syllable of a root. However, \phomtext{ə} generally will not recieve stress at this stage and move to the second syllable (even if this is also a schwa).

A subset of affixes (including, but not limited to those which utilize reduplication) are considered ``weak'', in that, like schwa, they repel stress.

Such morphemes will be marked with a comma in morphemic transcription, rather than the usual period used at syllable boundaries. Thus \morphtext{ʔə,} refers to a morpheme \phomtext{ʔə} which repels stress.

\subsection{Realization}
Under most circumstances, lexical stress is realized as a slight rise in pitch on the affected syllable, which then gradually falls back to ``standard'' pitch across the remaining part of the word. Because stress is very strongly initial, this means that generally, pitch will fall gradually across words and phrases.

This is of course also affected by phrase- and sentence-level prosody, which will be discussed in \Cref{ch:suprasegmental-phonology}.

\section{Allophony}
This section will be structured a bit differently from the others encountered thus far. Because allophony and diachronic processes in general occur in a particular order, Sectioning will be used to separate larger processes from those more easily described in standard notation.

While this document will cover some dialectical differences, examples given which include phonetic information Will be in reference to a standard dialect, \textbf{Literary Standard \langname}, Which will be abbreviated LSA

The ordering of rules will begin below, and these larger processes will be contained within subsections which will be referenced in the list.

\begin{enumerate}
  \item {
    Stressed \phomtext{ə} becomes \phontext{e}

    \phon{ə\phonfeat{+stress}}{e}
    }
  \item \titleref{sec:allophony-lengthening}
  \item {
    \phontext{ə} is elided between homorganic consonants

    \phonc{ə}{Ø}{\phonfeat{αcplace} \phold \phonfeat{αcplace}}
    }
    \item {
      \phontext{n} + plosive clusters cause place assimilation of the nasal

      \phonc{\phonfeat{+nasal}}{\phonfeat{αcplace}}{\phold \phonfeat{-cont}}
  }
    \item {
      Nasal + plosive clusters assimilate

      \phonc{\phonfeat{-cont}}{\phonfeat{+nasal}}{\phonfeat{+nasal} \phold}
  }

  \item {
      Unstressed final vowels are elided

      \phonc{V\phonfeat{-stress}}{Ø}{\phold \$}
  }
  % \aside{\fleuron\ If I'm being completely honest, I have no idea why the process of affrication is done with +delayed release. I'm simply following the conventions as outlined by \url{http://www.artoflanguageinvention.com/papers/features.pdf}}
  % \item {
  %     Word final geminated plosives are affricated

  %     \phonc{C\phonfeat{-cont \\ +long}}{C\phonfeat{+del release\footnotemark \\ -long}}{\phold \$}
  % }

\end{enumerate}

\subsection{Prosodic Lengthening}\label{sec:allophony-lengthening}
In some situations, syllable shape can cause gemination of the surrounding syllables--- prosodic lengthening, if you will, to allow for more consistent syllable-timing. The most notable of these situations is when a CV syllable, where C is a plosive, comes after a CVV sequence\footnotemark\ beginning in a non-plosive consonant.

\footnotetext{This is described as a CVV \textit{sequence} rather than a CVV syllable, because this process is not dependent on \phontext{VV} being a true diphthong. This can be seen in the first example, where \phontext{ua} Is not a diphthong but still triggers this change.}

Thus, pairs such as the following hypothetical forms may emerge:
\begin{itemize}
  \item \phomtext{suaki} \phontext{ˈsu.a.kːi} and \phomtext{suki} \phontext{ˈsu.ki}
  \item \phomtext{maipi} \phontext{ˈmai.pːi} and \phomtext{mapi} \phontext{ˈma.pi}
  \item \phomtext{iwəiku} \phontext{ˈi.we.kːu}\footnotemark\ and \phomtext{iwəku} \phontext{ˈi.wə.ku}
\end{itemize}

\footnotetext{This more aggressive simplification of diphthongs may give the appearance that plosives may geminate without an apparent trigger. Of course, this only applies to dialects which do not preserve the original diphthongs.}

\aside{\fleuron\ While moraic analysis of this process is not typical, one might suppose that this process is triggered by a heavy syllable followed by a light one, and that CVV is heavy, while VV is not.}
One thing of note is that the onset of the CVV syllable is mandatory to trigger this process. Observe the following hypothetical forms, Which do not undergo this change due to their VVCV shape:
\begin{itemize}
  \item \phomtext{auku} \phontext{ˈau.ku}
  \item \phomtext{aiʔi} \phontext{ˈai.ʔi}
  \item \phomtext{euti} \phontext{ˈo.ti}
\end{itemize}


% \chapter{Suprasegmental Phonology}\label{ch:suprasegmental-phonology}
On

\part{Morphology}
\chapter{Nominal}\label{ch:morpho-nom}
\section{Inflectional Morphology}\label{sec:morpho-nom-inf}
Inflectional morphology is very limited for nouns. The most marked nouns only inflect for case marking and plurality. However, depending on the context, plurality is optional. Depending on the constructions used, inflection of the noun may be even ungrammatical, where in other situations the same inflection would be entirely sensible.

\subsection{Case Marking}\label{sec:case-marking}
\langname 's case system is incredibly small. it consists solely of the cases used for morphosyntactic purposes; the ergative and absolutive cases. The ergative case is marked with reduplication of the initial syllable, while the absolutive case remains unmarked. This becomes slightly less transparent when this interacts with phonotactics, and when this reduplication triggers allophony.

\aside{\fleuron\ While phonological form is not the main focus of this section, the variation that can occur is such that I find it useful. It \textit{is} true that much of the prose simply reiterates what is found in the pronunciation information found in the gloss, however I consider this approach more useful for grasping the reasoning and triggers behind these phonological changes.}

With \notabletext{CV-initial forms,} this is quite straightforward:

\begin{subexamples}
  \baarucols{2}
  \ex
    \preamble pane
    \pronunciation ˈpa.nə
    \gloss
      pane & person[ABS] \\
  \ex
    \preamble papane
    \pronunciation pa.ˈpa.nə
    \gloss
      pa\allo pane & ERG\allo person \\
\end{subexamples}

\notabletext{V-initial forms} are slightly different, as adjacent vowels cannot be identical. An epenthetic \phomtext{ʔ} is inserted to avoid this.

\begin{subexamples}
  \baarucols{2}
  \ex
    \preamble ana
    \pronunciation ˈa.na
    \gloss
      ana & eye[ABS] \\
  \ex
    \preamble a'ana
    \pronunciation aʔ.ˈa.na
    \gloss
      aʔ\allo ana & ERG\allo eye \\
\end{subexamples}

\begin{subexamples}
  \baarucols{2}
  \ex
    \preamble ele
    \pronunciation ə.ˈle
    \gloss
      ele & heaven[ABS] \\
  \ex
    \preamble e'ele
    \pronunciation əʔ.ə.ˈle
    \gloss
      eʔ\allo ele & ERG\allo heaven \\
\end{subexamples}

\phomtext{ə} initial forms (such as the one above) are particularly interesting, in that they most clearly show an interaction between this process of reduplication and allophony. This reduplication can be described with the morpheme \morphtext{V,}. As a result \phomtext{əʔ.ə.lə} is phonetically \phontext{əʔ.ə.ˈle}, rather than \phontext{əʔ.ˈe.lə} as it would be with a standard morpheme.

\notabletext{CVV-initial forms} undergo a slightly different process. The reduplication in this situation does not even apply to the entire syllable, but rather only to the CV segment.

\begin{subexamples}
  \baarucols{2}
  \ex
    \preamble keuha
    \pronunciation ˈko.ha
    \gloss
      keuha & leaf[ABS] \\
  \ex
    \preamble kekeuha
    \pronunciation kə.ˈko.ha
    \gloss
      ke\allo keuha & ERG\allo leaf \\
\end{subexamples}

Consequently for LSA, the reduplicated vowel may be different than its phonetic realization in the stem due to an underlying diphthong.

\notabletext{VV-initial forms} work similarly to CVV forms, with the addition of an epenthetic \phomtext{ʔ} at the morpheme boundary:

\begin{subexamples}
  \baarucols{2}
  \ex
    \preamble auna
    \pronunciation ˈau.na
    \gloss
      auna & moon[ABS] \\
  \ex
    \preamble a'auna
    \pronunciation aʔ.ˈo.na
    \gloss
      aʔ\allo auna & ERG\allo moon \\
\end{subexamples}

As illustrated by the previous set of examples, this also triggers mono\-phthongization of the stem's initial syllable.

\subsection{Pluralization}\label{sec:pluralization}
Pluralization is very regularly marked through the infixation of \morphtext{me} between the reduplicated syllable of an ergative form and the stem. In absolutive forms, this gives the appearance that \morphtext{me} is a prefix.

Let's revisit some of the previous examples, now given in their plural forms:

\begin{subexamples}
  \baarucols{2}
  \ex
    \preamble mepane
    \pronunciation ˈpa.nə
    \gloss
      <me>pane & <PL>person[ABS] \\
  \ex
    \preamble pamepane
    \pronunciation pa.ˈme.pa.nə
    \gloss
      pa\allo <me>pane & ERG\allo <PL>person \\
\end{subexamples}

\begin{subexamples}
  \baarucols{2}
  \ex
    \preamble meana
    \pronunciation ˈme.a.na
    \gloss
      <me>ana & <PL>eye[ABS] \\
  \ex
    \preamble ameana
    \pronunciation a.ˈme.a.na
    \gloss
      aʔ\allo <me>ana & ERG\allo <PL>eye \\
\end{subexamples}


\begin{subexamples}
  \label{ex:double-schwa}
  \baarucols{2}
  \ex
  \preamble meele
  \pronunciation ˈme.lə
  \gloss
    <me>ele & <PL>heaven[ABS] \\
  \ex
    \label{ex:double-schwa-erg}
    \preamble emeele
    \pronunciation ə.ˈme.lə
    \gloss
      eʔ\allo <me>ele & ERG\allo <PL>heaven \\
\end{subexamples}

\aside{Respellings of loanwords from other languages commonly utilize \orthotext{ee}, although older generations generally dislike this emerging tendency, as some native words are being supplanted by these newer loans.}

\begin{subexamples}
  \baarucols{2}
  \ex
    \preamble mekeuha
    \pronunciation ˈme.ko.ha
    \gloss
      <me>keuha & <PL>leaf[ABS] \\
  \ex
    \preamble kemekeuha
    \pronunciation kə.ˈme.ko.ha
    \gloss
      ke\allo ke<me>uha & ERG\allo <PL>leaf \\
\end{subexamples}

\begin{subexamples}
  \baarucols{2}
  \ex
    \preamble meauna
    \pronunciation ˈme.au.na
    \gloss
      <me>auna & <PL>moon[ABS] \\
  \ex
    \preamble ameauna
    \pronunciation a.ˈme.o.na
    \gloss
      a\allo <me>auna & ERG\allo <PL>moon \\
\end{subexamples}

\baaruref{ex:double-schwa} is notable for being one of the few situations utilizing \orthotext{ee}. it is preserved in some words to distinguish otherwise opaque stress or as a result of etymology such as in \baaruref{ex:double-schwa-erg}.

\chapter{Adjectival}\label{ch:adjectives}
Adjectives are not a unique class of words in \langname .
What may look like ``adjectives'' on the surface are simply nouns.

\ex
\begingl
\glpreamble masaʔe si sixi
\pronounced{ˈma.sa.ʔə ˈsi ˈsi.xi}\endpreamble
\nogloss{\lbrack}
masa[sun]
ʔe[\textsc{assoc}]
si[brightness]
\nogloss{\rbrack}
sixi[\textsc{npst.ipfv}-shine]
\glft `The bright sun shines.'
\endgl
\xe

If the noun being modified in this way has ergative marking, it should be noted that the noun \textit{does not} inflect in agreement with the main noun.

\section{Adjective Ordering}

\langname 's\ basic adjective ordering is: «opinion» «size» «physical quality» «shape» «age» «color» «origin» «material» «type» «purpose»

\section[\langword{ʔe} Ellipsis]{NP-Ellipsis with \langword{ʔe} (\langword{ʔe} Ellipsis)}\label{sec:ellipsis}

In some cases however, this basic ordering may be deviated from.
A single adjective may be placed before \langword{ʔe,} allowing the main noun itself to be dropped, and the main to be referenced in futher discourse using the promoted adjective + ʔe in a form of NP-ellipsis.
This is particularly useful when many of the same object with similar but differing qualities are being discussed for extended lengths of time (for example, a discussion about two different people, or about several types of a similar object).
It may also be used, as seen below, to chain clauses together.

This ellipsis is not entirely dissimilar to the function of ``That X thing'', though the construction itself is quite different.

\ex<optional_ellipsis>
\begingl
\glpreamble masa siʔe sixi, siʔe kaʔa
\pronounced{ˈma.sa ˈsi.ʔe ˈsi.xi | ˈsi.ʔe ˈka.ʔa}\endpreamble
\nogloss{\lbrack}
masa[sun]
si[brightness]
ʔe[\textsc{assoc}]
\nogloss{\rbrack}
sixi[\textsc{npst.ipfv}-shine]
si-ʔe[brightness-ʔe]
kaʔa[happiness]
\glft `The bright sun shines, and its light makes me happy.'
\endgl
\xe

It should also be noted that this ellipsis is often not the only way sentences which utilize it could be expressed.
In cases such as that of Example \getfullref{optional_ellipsis}, ellipsis is purely a stylistic preference, especially when used isolated from other context.
In particular, \getfullref{optional_ellipsis} may also be expressed as:

\ex
\begingl
\glpreamble masaʔe si sixi, siʔe kaʔa
\pronounced{ˈma.sa.ʔe ˈsi ˈsi.xi | ˈta ˈka.ʔa}\endpreamble
\nogloss{\lbrack}
masa[\textsc{sun}]
ʔe[\textsc{assoc}]
si[brightness]
\nogloss{\rbrack}
sixi[\textsc{npst.ipfv}-shine]
ta[\textsc{prox}]
kaʔa[happiness]
\glft `The bright sun shines, and its light makes me happy.'
\endgl
\xe

Here, the same information is presented, but in a slightly different manner.
\textit{ʔe-Ellipsis,} as it will now be referred to as, is not utilized.
Default adjective order is returned to, and the proximate \langword{ta} is used resumptively in reference to \langword{masa.}

The necessity of \langword{ta} here is dependent on the speaker, as \langword{sixi} is strictly intransitive and the ommission of \langword{ta} may not impact intelligibility for this group of speakers, where context is suffucuent in maintaining understanding.
%TODO: Cover examples with many adjectives
\chapter{Verbal}

In \langname , verb inflection is extremely minimal.
Most verbal infornation is conveyed through peripherastic constructions, the most common being multiple-verb constructions.

\begin{table}[ht]
  \centering
  \begin{tabular}{*{3}{c}}
    \toprule
                 & Nonpast & Past  \\\midrule
    Imperfective & ∅-      & i-    \\
    Perfective   & V(ʔ)-   & m(u)- \\
    \bottomrule
  \end{tabular}
  \caption{Verb Inflection}
  \label{table:verb}
\end{table}

The \textsc{npst.pfv} form is special.
The vowel inserted is dependent on the nucleus of the syllable it attaches to.
V `echoes' from adjacent syllables.
Thus, a form such as \langword{ama,} conjugated in the \textsc{npst.pfv} form, becomes \langword{aʔama.}

Similarly, a form with an initial consonant such as \langword{litu} will have its nucleus echoed to produce \langword{ilitu} as the \textsc{npst.pfv} form.
But it also brings up the question, how does this differ from the \textsc{pst.ifv} form? This echo vowel behaves slightly differently than a typical \langname\ prefix.
It is unable to be stressed, which in turn means that \langword{litu} will be realized as [ˈi.li.tu] in the \textsc{pst.ipfv} form, but [iˈli.tu] in the \textsc{npst.pfv} form.
These minimal pairs only occur with i-stem verbs.

\section{Alignment}

Verbs follow direct-inverse alignment, which utilizes a person hierarchy to determine the appropriate verb marking.
The direct construction is used when the agent of the transitive clause outranks the patient in the person hierarchy, and the inverse is used when the patient outranks the agent.
It should be made clear that this type of alignment coexists with the ergative-absolutive alignment of nouns.

\subsection{Person Hierarchy}\label{sec:person_hierarchy}
One of the core mechanisms of a direct-inverse system is its person hierarchy.
As mentioned above, this determines if the verb will be used in the direct or inverse form, based on the relative positions of verbal arguments in the hierarchy.

The person hierarchy in \langname\ is: 2\textsuperscript{nd} person > 1\textsuperscript{st} person > 3\textsuperscript{rd} person proximate (\textsc{prox}) > 3\textsuperscript{rd} person obviative (\textsc{obv}).

The proximate/obviative distinction is particularly notable.
It is used to disambiguate situations where both arguments of the verb are 3\textsuperscript{rd} person.
It is not uncommon for the proximate/obviate marking to be dropped, due to functional overlap with the noun cases that \textit{do} exist.

\pex<alignment-dir>
% \a<grammatical>
\begingl
\glpreamble tutu kula me
\pronounced{ˈtu.tu ˈku.la ˈme}\endpreamble
tu\textasciitilde tu[\textsc{erg\textasciitilde 2sg}]
kula-∅[hurt-\textsc{dir}]
∅-me[\textsc{abs-1sg}]
\glft `You are hurting me.'
\endgl
\a<ungrammatical>
\begingl
\glpreamble \ljudge{*} du kula meme
\endpreamble
∅-tu[\textsc{abs-2sg}]
kula-∅[hurt-\textsc{dir}]
me\textasciitilde me[\textsc{erg\textasciitilde 1sg}]
\endgl
\xe

\pex<alignment-inv>
\a<grammatical>
\begingl
\glpreamble meme kulasi tu
\pronounced{ˈme.mə ˈku.la.si ˈtu}\endpreamble
me-me[\textsc{erg\textasciitilde 1sg}]
kula-si[hurt-\textsc{inv}]
∅-tu[\textsc{abs-2sg}]
\glft `I am hurting you.'
\endgl
\a<ungrammatical>
\begingl
\glpreamble \ljudge{*} me kulasi tutu
\endpreamble
me[\textsc{abs-1sg}]
kula-si[hurt-\textsc{inv}]
tu-tu[\textsc{erg\textasciitilde 2sg}]
\endgl
\xe

The latter parts of these examples are ungrammatical, because the case marking on the nouns implies roles which are opposed by the marking on the verbs.

The direct-inverse marking is not entirely redundant, as it may seem from the above examples.
The overlap of these systems allows some pronouns to be dropped, with no loss in clarity.
For example, Example \getfullref{alignment-dir.grammatical} may also be conveyed as follows:

\ex
\begingl
\glpreamble kula me
\pronounced{ˈku.la ˈme}\endpreamble
kula-∅[hurt-\textsc{dir}]
∅-me[\textsc{abs-1sg}]
\glft `You are hurting me.'
\endgl
\xe

There is no explicitly stated agent in this example, but it is not necessary in this instance.
The verb \langword{kula} is in its direct form, meaning its agent outranks its patient in the person hierarchy.
Only one thing outranks a 1\textsuperscript{st} person patient, that being a 2\textsuperscript{nd} person agent.

In a similar fashion, the object of an obviously transitive verb may be dropped as well:

\ex
\begingl
\glpreamble kula!
\pronounced{ˈku.la}\endpreamble
kula-∅[hurt-\textsc{dir}]
\glft `You are hurting me.'\\`It hurts!'
\endgl
\xe

Because of the origin of this construction, this may not be used in the same general way `It hurts!' can be in English, though \langword{kula!} is still translated as such based on context.
Natural phenomena, for example, can be reeferenced in this manner of exclamation, as they are still seen as somewhat animate.
This is the mechanism by which the following functions:
\ex
\begingl
\glpreamble kula! aʔahisiʔe tetaʔa, ta lixi nasu ɸa me\endpreamble
kula[\textsc{intj}]
aʔahisi[\textsc{erg:}wind]
ʔe[\textsc{assoc}]
tetaʔa[cold]
ta[prox]
lixi[freeze]
nasu[nose]
ɸa[\textsc{poss}]
me[\textsc{abs:1sg}]
\glft `Ouch! That cold wind freezes my nose!'
\endgl
\xe

\section{Tense and Aspect}
\subsection{Constructing the Future}

\langname\ makes no morphological distinction between present and future tense.
In everyday discourse, its speakers avoid explicitly referring to the future, as it is seen as overly speculative for most purposes.
This is especially noticable when the reoccurence of a habitual action is questioned.
To exemplify this with an English example, a speaker of \langname\ will prefer ``The sun rises?'' to a similar ``Will the sun rise?''.
Self evident truths such as these do not require additional tense marking, as the interrogative fills the same role.

\subsubsection{Habitual Imperfectives}

The above question and and answer would be rendered as follows in \langname :

\ex<ex:future_interrrogative>
\begingl
\glpreamble ti masa
\pronounced{ˈti ˈma.sa}\endpreamble
∅-ti[\textsc{npst.ipfv-}rise\textsc{[q]}]
masa[sun]
\glft `The sun rises?'\\`Will the sun rise?'
\endgl
\xe

Note that even though this event may be typically rendered with the perfective aspect, the imperfective is used instead.
This is because the rising of the sun is known to be habitual, and so the imperfective is used to show the knowledge of this.
This is not typical, but is commonly done when referring to things that happen out of human control, thimgs which are ``just the way the world works''.
Phenomena which follow this priciple include:

\begin{itemize}
  \item The passage of seasons
  \item Cycles of the sun and moon
  \item Other cyclic natural processes
  \item Time, in the context of inevitability and continual change
\end{itemize}

Now that the question has been constructed, we must construct an answer.
The SVO order, typical of declarative sentences, is returned to.
In a simple example such as this one, the rest of the sentence remains unaltered.
And thus, an \langname\ speaker will simply reply:

\ex<ex:implied_future>
\begingl
\glpreamble masa ti
\pronounced{ˈma.sa ˈti}\endpreamble
masa[sun] ∅-ti[\textsc{npst.ipfv-}rise]
\glft `The sun rises.'
\endgl
\xe

These imperfectives continue to be read as specifically habitual, even in declarative sentences.
To convey a progressive reading, a duration must be specified.
It should also be noted that units of time have a implicit quantity of one, and so no explicit mention of quantity of hours is necessary in this example.

\ex
\begingl
\glpreamble masa neʔe ti
\pronounced{ˈma.sa ˈne.ʔə ˈti}\endpreamble
masa[sun]
neʔe[hour]
∅-ti[\textsc{npst.ipfv-}rise]
\glft `The sun has been rising for one hour.'
\endgl
\xe

\ex
\begingl
\glpreamble masa pa ti
\pronounced{ˈma.sa ˈpa ˈti}\endpreamble
masa[sun]
pa[now]
∅-ti[\textsc{npst.ipfv-}rise]
\glft `The sun is rising.'
\endgl
\xe

\subsubsection{Explicit Future}
On the rare occasion that the future must be explicitly marked, a different construction can be used.
\langname\ has as expression which loosely translates to ``It was, it is [therefore it must always be]''.
This implied segment allows similar constructions to be used in a grammatical context, as well as in discourse.

This construction can be used with almost any verb, but it is avoided due to its formality, and doubly so because an archaic copula, \langword{wa.}

\ex
\begingl
\glpreamble iwa, alete wa
\pronounced{ˈi.wa, ˈa.lə.te ˈwa}\endpreamble
i-wa,[\textsc{pst.ipfv-cop}]
alete[thus]
∅-wa[\textsc{npst.ipfv-cop}]
\glft `It was, it is, [therefore it must always be].'
\endgl
\xe

The copula used in this phrase is another of these verbs, as it refers more to the ongoing passage of time rather than the event itself (if it were, \langword{wa} would not be used in this fashion).

Here, Example \getfullref{ex:implied_future} is rendered explicitly in the future, using \textit{wa-repetition.}
\ex<ex:explicit_future>
\begingl
\glpreamble masa iti, alete ti
\pronounced{ˈma.sa ˈi.ti | ˈa.lə.te ˈti}\endpreamble
masa[sun]
i-ti[\textsc{pst.ipfv-}rise,]
alete[thus]
∅-ti[\textsc{npst.ipfv-}rise]
\glft `The sun will rise.'
\endgl
\xe

The most notable exception is when discussing emotion, in which case this construction cannot be used at all.

\ex<ex:emotion_wrong>
\begingl
\glpreamble \ljudge{*} inale, alete nale
\pronounced{ˈi.na.le | ˈa.lə.te ˈna.lə}
\endpreamble
i-nale,[\textsc{pst.ipfv-}be\_sad]
alete[thus]
∅-nale[\textsc{npst.ipfv-}be\_sad]
\glft `I will be sad.'
\endgl
\xe

Due to the sense of permanence the construction creates, it is ungrammatical to use here.
It may even sound to some to be a malformed causative, but this is neither grammatically nor semantically correct.
Emotions in \langname\ are treated as only things felt.
Although a person can \textit{do things} to cause an emotional response, they cannot cause them directly.
This idea is reflected in the constructions used in reference to emotion.

The proper way to convey Example \getfullref{ex:emotion_wrong} is shown here:

The usual strategy of \textit{wa-repetition} for dynamic verbs is not applicable for stative verbs, and as such a different strategy must be used.
Stative verbs can only be explicitly placed into the future tense through the use of an auxilliary verb, \langword{sihu} placed before the main verb.

\ex
\begingl
\glpreamble sihu nale
\pronounced{ˈsi.hu ˈna.lə}\endpreamble
Ø-sihu[\textsc{npst.ipfv-}happen]
Ø-nale[\textsc{npst.ipfv-}be\_sad]
\glft `I will be sad.'
\endgl
\xe

In discourse, this construction (from now on referred to as \textit{wa-repetition}) has a similar, but distinct meaning to its grammatical counterpart.
Any one of ``So be it.'', ``It is what it is.'', or sometimes even ``Leave it be.'' may be apt translations.
In these situations, the phrase may be shortened to `iwalete', although this is seen as incredibly informal and potentially rude.

\section{Mood}
\subsection{Expressing Interrogatives}

\langname 's interrogative is expressed through a change in word order.
The verb is fronted, and conjugated as would be expected.

\ex
\begingl
\glpreamble muti xanu
\pronounced{ˈmu.ti ˈxa.nu}\endpreamble
mu-ti[\textsc{pst.pfv-}rise\textsc{[q]}]
xanu[bear]
\glft `Did the bear get up?'
\endgl
\xe

The type of question being asked is often left up to context, and is usually translated using English `do' or `how' when there is insufficent context to make a finer distinction.
Ambiguous questions, as seen here, also form polar questions.
Deliberatly underspecifying these is done as well.

\ex
\begingl
\glpreamble mai muti xanu
\pronounced{ˈmai ˈmu.ti ˈxa.nu}\endpreamble
mai[time]
mu-ti[\textsc{pst.pfv-}rise\textsc{[q]}]
xanu[bear]
\glft `When did the bear get up?'
\endgl
\xe

Several things should be noted from this example.
\langname\ does not have dedicated wh-forms, instead using nouns for this purpose, which behave somewhat like an adverb, specifying what specifically is being questioned.
he words used in this clarification can function as typical nouns as well.
The use of these nouns is not mandatory, however, as the nature of the question can sometimes be inferred through context.
When nouns are used as clarification of interrogatives, they are placed directly before the verb, and left unmarked for case.
This is significant in that \langword{muti} is being used intransitively, rather than in a transitive sense with \langword{mai} as an argument of the verb.

\ex
\begingl
\glpreamble mamai uhumi xaku
\pronounced{ˈma.mai uˈɣu.mi ˈxa.ku}\endpreamble
ma\textasciitilde mai[\textsc{erg\textasciitilde}time]
u-humi[\textsc{npst.pfv-}heal]
∅-xaku[\textsc{abs-}pain]
\glft `Time heals all wounds.'
\endgl
\xe
In this example, \langword{uhumi} `to heal', has \langword{mai} as an argument.
It is being used transitively, as evidenced by the direct object \langword{xaxaku} and the presence of ergative marking.
Though \langword{mai} precedes the verb, there is no interrogative reading of this example.

Many words can be used to clarify the nature of an interrogative, far more than the interrogative pronouns of English.
Here are just a few examples of how this can be used:

\ex
\begingl
\glpreamble laʔia xanu
\pronounced{ˈlaʔia ˈxa.nu}\endpreamble
laʔia[color]
xanu[bear]
\glft `What color is the bear?'
\endgl
\xe

\ex
\begingl
\glpreamble kane xanu
\pronounced{ˈka.nə ˈxa.nu}\endpreamble
kane[strength]
xanu[bear]
\glft `How strong is the bear?'
\endgl
\xe

\subsection{Imperative Construction}

While imperatives are not strictly formed through morphological means, the specific considerations to be made in the application of existing morphology to achieve this has made the inclusion of that process here a sensible one.

\subsubsection{Transitives}
\ex
\begingl
\glpreamble meme ti ta
\pronounced{ˈme.mə ˈti ˈta}\endpreamble
me\textasciitilde me[\textsc{erg\textasciitilde1sg}]
ti[rise]
ta[\textsc{prox}]
\glft `I get (myself) up.'
\endgl
\xe

\langword{Ti} is a verb of posture, and in this case can be used to orient sequences of events.
When used in this way, it imparts a meaning similar to `to prepare oneself'.
Compare this to the nonreflexive meaning simply `I rise', and the more idiomatic `I am here'.

This is important, because these postural verbs assist to frame imperatives.
By additionally using an SVC, imperatives are described as happening to the speaker, but shifted onto the one who is intended to follow the command.
This will be illustrated in the following examples.

\ex
\begingl
\glpreamble meme ti ihesu ta lu tu
\pronounced{ˈme.mə ˈti ˈi.hə.su ˈta ˈlu ˈtu}\endpreamble
me\textasciitilde me[\textsc{erg\textasciitilde1sg}]
ti[rise]
ihesu[ihesu]
ta[\textsc{prox}]
lu[\textsc{obv}]
tu[\textsc{2sg}]
\glft `Get up!'
\endgl
\xe

In this example, \langword{ti} is used both with its postural meaning, and its more standard interpretation.
That is to say, that every imperative is framed with a postural verb like this, each imparting slightly different urgencies and expectations.

\langword{Ti} is often considered the most neutral of the postural verbs in the context of imperatives, because it simply implies that something will happen soon, that something being the command.


\ex
\begingl
\glpreamble meme ti ihesu ta lu tu
\endpreamble
me\textasciitilde me[\textsc{erg\textasciitilde1sg}]
ti[rise]
palaʔani[curl]
ihesu[ihesu]
ta[\textsc{prox}]
lu[\textsc{obv}]
tu[\textsc{2sg}]
\glft `Get up! (sometime soon, but not too lenient or it will be more egregious than simply refusing to begin with)'
\endgl
\xe

\ex
\begingl
\glpreamble meme ti ihesu ta lu tu
\endpreamble
me\textasciitilde me[\textsc{erg\textasciitilde1sg}]
ti[rise]
laku[swim]
palaʔani[curl]
ihesu[ihesu]
ta[\textsc{prox}]
lu[\textsc{obv}]
tu[\textsc{2sg}]
\glft `Get up (definitely) and swim if you feel like it.'
\endgl
\xe

\ex
\begingl
\glpreamble meme ti ihesu ta lu tu
\endpreamble
me\textasciitilde me[\textsc{erg\textasciitilde1sg}]
sulau[push\_open]
???[chair]
ti[rise]
ihesu[ihesu]
ta[\textsc{prox}]
lu[\textsc{obv}]
tu[\textsc{2sg}]
\glft `Go seat yourself.'
\endgl
\xe

\part{Syntax}

\chapter{Nominal}

\section{Adpositions}

\begin{description}
  \item[\langword{wu}:] On top (of), from above
  \item[\langword{he}:] Under, from below
  \item[\langword{sa}:] Outside of, from outside
  \item[\langword{i}:] In, from inside
  \item[\langword{ana}:] Next to, near
\end{description}

\ex
\begingl
\glpreamble meme sulau mesikima he line
\pronounced{ˈme.mə ˈsu.lau ˈme.si.ki.ma ˈɣe ˈli.nə}\endpreamble
meme[\textsc{1sg:erg}]
sulau[shoo]
mesikima[\textsc{pl:}fly]
he[under]
line[light]
\glft `I am shooing the flies out from under the light'
\endgl
\xe

This example, in regards to the preposition itself, is has ambiguous directionality, in that the flies may be shooed either towards or away from the light. Generally this must be inferred, explicit marking of this is uncommon. In the case of \langword{sulau} however, the latter meaning is the only sensible reading. A similar but opposite effect occurs with \langword{kamai} `to send', which is unabiguously read with the former meaning, illustrated below:

\ex
\begingl
\glpreamble mamemalaku ikamai he sesiti
\pronounced{ˈma.mə.ma.la.ku ˈi.ka.mai ˈɣe ˈse.si.ti}\endpreamble
mamemaleku[\textsc{erg:pl:}cat]
ikamai[\textsc{pst.pfv:}send]
he[under]
sesiti[blanket]
\glft `The cats hid under the blanket'
\endgl
\xe

The verb \langword{kamai} is strictly transitive, so \langword{ta} can be used resumptively to form what is essentially a reflexive construction. Because \langword{ta} does not precede a verb, it also cannot be read as a labile verb.

\ex
\begingl
\glpreamble me ɸene wu ehaixa
\pronounced{ˈme ˈɸe.nə ˈwu ˈe.hai.xa}\endpreamble
me[\textsc{abs:1sg}]
ɸene[live]
wu[on\_top]
ehaixa[\textsc{name}]
\glft `I live in Ehaixa'
\endgl
\xe

Prepositions are not always translated with their standard definitions. For example, with names of proper nouns, \langword{wu} `on top (of)' is used to express residency. Whereas:

\ex
\begingl
\glpreamble me ɸene i pihane
\pronounced{ˈme ˈɸe.nə ˈi ˈpi.ha.nə}\endpreamble
me[\textsc{abs:1sg}]
ɸene[live]
i[inside]
pihane[house]
\glft `I live in a house'
\endgl
\xe
This sentence is translated with \langword{i} `inside', using the more typical reading.

A similar dichotomy can be found between \langword{i} and \langword{wu} when in reference to water, particularly its depth. So one might say:

\ex
\begingl
\glpreamble me mulaku i nauwa
\pronounced{ˈme ˈmu.la.ku ˈi ˈnau.wa}\endpreamble
me[\textsc{abs:1sg}]
mulaku[pst.pfv:swin]
i[inside]
nauwa[water]
\glft `I swam underwater'
\endgl
\xe

This meaning the core argument, \langword{me,} was fully encompassed (ie. submerged at a considerable depth) by the water.

\ex
\begingl
\glpreamble me mulaku wu nauwa
\pronounced{ˈme ˈmu.la.ku ˈwu ˈnau.wa}\endpreamble
me[\textsc{abs:1sg}]
mulaku[pst.pfv:swin]
wu[on\_top]
nauwa[water]
\glft `I swam in the water'
\endgl
\xe

Whereas here, there is no indication of considerable depth. It should be noted that simply swimming in a deep body of water does not warrant the use of \langword{i,} the depth of water must be relevant to the situation at hand.

\chapter{Verbal}

\section{Focus}

\ex
\begingl
\glpreamble ɸumau keke hahawi
\pronounced{ˈɸu.mau ˈke.kə ˈɣa.ha.wi}\endpreamble
Ø-ɸumau[\textsc{abs-}grass]
Ø-keke[\textsc{npst.ipfv-}eat]
ha\textasciitilde hawi[\textsc{erg\textasciitilde}rabbit]
\glft `The grass is being eaten by the rabbit'
\endgl
\xe

Because the grass is still the patient of the verb, it is still marked with the ergative. Fronted arguments of transitive verbs become focused. A passive construction will be used in translation to English. This is solely to approximate the topicalization, as this example is not a true passive (the verb's valency is not decreased). Arguments in default position can be focused, albeit in a different manner. Returning to Example \getfullref{alignment.trns}, but with the agent explicitly focused:

\ex
\begingl
\glpreamble hahawi keke lu ɸumau
\pronounced{ˈɣa.ha.wi ˈke.kə ˈlu ˈɸu.mau}\endpreamble
ha\textasciitilde hawi[\textsc{erg\textasciitilde}rabbit]
keke[\textsc{npst.ipfv-}eat]
lu[\textsc{obv}]
ɸumau[\textsc{abs,}grass]
\glft `The rabbit (as opposed to something else) is eating the grass'
\endgl
\xe

By marking the already established patient with the obviative\footnotemark, it puts more focus on the (unmarked) proximal argument than would be typical.

\footnotetext{The standard use of proximate/obviate morphology is falling out of use in favor of case marking, Remaining instances have either become fossilized in expressions and idioms, or fulfilled another grammatical purpose, as seen here.}

\section{Labile Verbs}\label{sec:labile_verbs}

A labile verb is a verb that can be either transitive or intransitive, and whose subject when intransitive corresponds to its direct object when transitive. They are also sometimes referred to as ``S=O ambitransitive`` verbs. A prototypical example of this being ``John tripped'' in contrast with ``John tripped Tim''. Unlike a typical ambitransitive verb, the subject's role changes.

\ex
\begingl
\glpreamble ɸihaʔau me
\pronounced{ˈɸi.ha.ʔo ˈme}\endpreamble
ɸihaʔau-∅[trip\textsc{-dir}]
me[\textsc{1sg.abs}]
\glft `You tripped me'
\endgl
\xe

\ex
\begingl
\glpreamble me ɸiahau
\pronounced{ˈme ˈɸi.a.ho}\endpreamble
me[\textsc{1sg.abs}]
ɸiahau-∅[trip-\textsc{-dir}]
\glft `(You) tripped me'
\endgl
\xe

These first two examples utilize concepts which have previously been covered in \Sref{sec:person_hierarchy}. The following utilizes the proximate particle, \langword{ta,} in order to mark \langword{me} as the most agentlike argument of a transitive verb. As such, there is no possibility of inferring an agent of \langword{ɸihaʔau.} In this way, labile verbs can be expressed without the need for a dummy agent.

\ex
\begingl
\glpreamble ta me ɸihaʔau
\pronounced{ˈta ˈme ˈɸi.ha.ʔo}\endpreamble
ta[\textsc{prox}]
me[\textsc{1sg.abs}]
ɸihaʔau-∅[trip\textsc{-dir}]
\glft `I tripped'
\endgl
\xe


\chapter{Discourse}

\section{Discourse Repair}
As an example, lets imagine a hypothetical speaker just said the following:
\ex
\begingl
\glpreamble hahawi keke lu ɸumau
\pronounced{ˈɣa.ha.wi ˈke.kə ˈlu ˈɸu.mau}\endpreamble
ha\textasciitilde hawi[\textsc{erg\textasciitilde}rabbit]
keke[\textsc{npst.ipfv-}eat]
lu[\textsc{obv}]
ɸumau[\textsc{abs,}grass]
\glft `The rabbit (as opposed to something else) is eating the grass'
\endgl
\xe

\subsection{Nominals}

If someone mishears, or for whatever reason needs clarification on the arguments of a transitive verb, the obviative and proximate markers can be used.

\begin{paracol}{2}
If the listener only hears \langword{hahawi ke\-ke lu}, and not the \textit{patient}, the listener can ask the following:
\ex
\begingl
\glpreamble lu?
\pronounced{ˈlu}\endpreamble
lu[\textsc{obv}]
\glft `Eating what?'
\endgl
\xe
\switchcolumn

If a listener only hears \langword{keke lu ɸumau,} and not the \textit{agent}, the listener can ask the following:

\ex
\begingl
\glpreamble ta?
\pronounced{ˈta}\endpreamble
ta[\textsc{prox}]
\glft `What is eating grass?'
\endgl
\xe
\end{paracol}

\subsection{Verbal}
If our listener only heard \langword{``hahawi --- lu ɸumau'',} the response may be:
\ex
\begingl
\glpreamble ta sihu lu?
\pronounced{ˈta ˈsi.hu ˈlu}\endpreamble
ta[\textsc{prox}]
sihu[happen]
lu[\textsc{obv}]
\glft `The rabbit is doing what to grass?'
\endgl
\xe

\langword{ta} and \langword{lu} are used here in a resumptive fashion, rather than repeating the content words. This implies more confidence, in that repeating \langword{hahawi} or \langword{ɸumau} may imply that the listener is also unsure of these components as well, rather than just the verb.

Because this phrase is somewhat of a standard one, it is shortened in colloquial speech. The most aggresive of these shortenings being [ˈtasul(ə)].

\subsection{Responding to Repair Questions}\label{sec:repair_response}

Repair questions can be responded to quite similarly to how a ``standard'' question would be. The main difference is the necessity of the associative particle, \langword{ʔe} to connect \langword{ta} or \langword{lu} to the appropriate content word(s). For example:

\ex
\begingl
\glpreamble ta ʔehawi
\pronounced{ˈta ˈʔe.ha.wi}\endpreamble
ta[\textsc{prox}]
ʔe[\textsc{assoc}]
hawi[rabbit]
\glft `The rabbit (is eating grass)'
\endgl
\xe
\subsection{Self-Correction}
\subsubsection{In conjunction with \langword{ʔe} Ellipsis}

The same structures used to respond to repair questions may be employed to correct or clarify the ellipsed NP, for example if the remaining information  is too ambiguous, incorrect, or is simply no longer relevant.

\ex<adjective_correction>
\begingl
\glpreamble tatakaʔe ala ta ʔekatu kulasi me
\pronounced{ˈta.ta.ka.ʔə ˈa.la ˈta ˈʔe.ka.tu ˈku.la.si ˈme}\endpreamble
tataka[\textsc{erg:}rock]
ʔe[\textsc{assoc}]
ala[white]
ta[\textsc{prox}]
ʔe-katu[ʔe-sharpness]
kulasi[hurt\textsc{:inv}]
me[\textsc{abs:1sg}]
\glft `The big white --- no, sharp --- rock is hurting me'
\endgl
\xe

When \langword{ʔe} is used in discourse repair, it attaches to the correction, rather than to \langword{ta} or \langword{lu.} This the the opposite of when ʔe is used associatively (see \Sref{ch:adjectives}), where \langword{ʔe} attaches to the noun being described.

\subsubsection{New Adjectives}
Similarly, if a noun which previously had no adjectives modifying it needs to have one added in the middle of discourse, the following can be done:

\ex
\begingl
\glpreamble ta nene xase lu ʔe kela
\pronounced{ˈˈta ˈne.nə ˈxa.sə ˈlu.ʔe ˈke.la}\endpreamble
ta[\textsc{prox}]
nene[paint]
xase[spill]
lu[\textsc{obv}]
ʔe[\textsc{assoc}]
kela[green]
\glft `The paint spilled, it's green'
\endgl
\xe

%? Does green exist as its own basic term? Decide.

Special attention should be given when referencing arguments of labile verbs (see \Sref{sec:labile_verbs}). While \langword{ta} marks \langword{nene} as the most agent-like, it is still morphologically and syntactically the verb's patient (note the lack of ergative marking). Because of this, core (S) arguments of labile verbs should be used with the resumptive pronoun used with patients, \langword{lu.}

Also, adjectival additions such as this are strongly preferred to be done \textit{after} the verb, particularly with intransitive verbs. These types of corrections are done differently in that the adjective \textit{is} associated to \langword{lu}. This is different from correcting the use of an incorrect adjective, seen in Example \getfullref{adjective_correction}.

\part{Semantics/Pragmatics}
\chapter{People}
\section{Kinship}

\subsection{Classification}
\langname\ can be broadly categorized as having a Hawaiian kinship system, wherein cousins share the same terms as siblings, and aunt/uncle do not have distinct terminology from parents. In a traditional Hawaiian kinship system, gender is distinguished, but this is not the case in \langname .

Age is used as the secondary division of kinship terminology, wherein younger sibling/older sibling are each a distinct lexeme, rather than being formed through adjectives or other derivational means. The same can be said of parent/aunt/uncle, which share one term, aside from division by age.

The following table summarizes a few of the most basic terms used within \langname 's kinship system. Note that whenever possible, native terms will be used to avoid clumsy explanation.

\begin{table}[ht]
  \centering
  \begin{tabular}{lll}
    \toprule
                      & Same/Younger      & Older           \\ \midrule
    Child             & \langword{ausuʔi} & \langword{teme} \\
    Sibling           & \langword{naʔuwe} & \langword{wase} \\
    Parent/Aunt/Uncle & \langword{kausu}  & \langword{mele} \\ \bottomrule
  \end{tabular}
  \caption{Basic Kinship Terminology}
\end{table}

\subsection{Age Distinction}
\langname 's age distinction is based on relative age of other members within the generation. That is to say, if one's mother is younger than one's father, `young-parent' will be used for her, and vice-versa.

If the people involved were an aunt (who is younger than the respective uncle), and the mother (also younger than the respective father) of the ego,\footnotemark\ things become slightly ambiguous, without further explanation: Are both referred to as `young-parent', or are their ages compared as well?

The former is indeed the case. Consider the following examples in relation to an ego with two \langword{naʔuwe} and one \langword{wase}:

\ex<kin_inv>
\begingl
naʔuwe[young\_sibling]
kulasi[hurt\textsc{-inv}]
naʔuwe[young\_sibling]
\glft `My (younger) sibling is hitting their (older) sibling.'
\endgl
\xe

\footnotetext{The ego is the person from which kinship terms are arranged relative to.}

This raises a few points of interest:
\begin{itemize}
  \item Unless specified within a conversation, kinship terms are always used with the expectation that the ego is the \textit{speaker,} rather than the subject
  \item Verbal morphology can be indicative of further granularity which is not evident solely from the terms themselves
\end{itemize}

The first point is relatively evident, made so by the use of \langword{naʔuwe} for both referents, coupled with the translation. This tells us a lot, in fact, based upon the premise established above.

Firstly, we can deduce that the \langword{naʔuwe} which is the subject is the youngest of the ego and between the \langword{menaʔuwe} and \langword{me.} This is because of the inverse marking, which allows the observation that the first \langword{naʔuwe} is younger than the second. The use of \langword{naʔuwe} itself informs us of the relationship to the ego in such a situation.

Also, unlike most other nouns, kinship terms are assumed to be implicitly related to each other, hence the above does not refer to a \langword{naʔuwe} who hits a \langword{naʔuwe} from another family.

\subsection{Utterance Parroting with \langword{Ihesu}}
The strategy used above is fragile, since the speaker must reframe situations in which they are not the ego, and this is not always desirable. A special verb, \langword{ihesu,} can be used to quote, or `parrot' the speech of another person, in which case the ego shifts to the quoted person rather than the speaker as normally would be the case. This should not be used in all situations, as \langword{hanu} is often enough when simply repeating a phrase, whereas \langword{ihesu} serves to shift perspective completely.

\ex<parroting>
\begingl
tutu[\textsc{erg\textasciitilde 2sg}]
mukula[\textsc{pst.pfv-}hurt]
ihesu[ihesu]
lu[lu]
ausuʔi[young\_child]
kausu[young\_parent]
\glft `\,``You hurt my child!'', is what my parent said.'
\endgl
\xe

In situations where both \langword{ihesu} and \langword{hanu} could be used, \langword{ihesu} is preferred for formal purposes, such as when a courier delivers a message, speaking for the person having sent it, voicing or `parroting' --- as I've used here, the message for the original sender. Whereas \langword{hanu} still requires the original speaker as the agent, and has slightly more reportative connotations.

\subsection{Morphosyntactic Implications}
Kinship terms, and occassionally the names of specific people, can change how a sentence must be constructed. At the most basic level, these terms and their relationships within a phrase are of direct consequence to the alignment of the verbs they are used in. Older generations are seen as higher on the person hierarchy (see \Sref{sec:person_hierarchy} for general guidelines about this), and therefore will have affect on verbal morphology.

It should be noted though that this is not applicable when the referents are of different families. That is to say, a grandparent of hypothetical family A will never outrank a parent from hypothetical family B (or any other family member). These considerations should only be made from within a single family.

From within a generation, all members are on the same level of the person hierarchy (except for exceptions as with Example \getfullref{kin_inv}), and as such will use the form of the verb as would be done ordinarily.


\section{Formality}
\chapter{Perception}
\section{Color}
\section{Time}
\section{Natural Forces}

\part{Appendices}
\appendix
\chapter{Additional Example Sentences}
\ex
\begingl
\glpreamble mapiʔe ai pa tasaʔe kiɸe tamuka asuhi
\pronounced{ˈma.pi.ʔə ˈai ˈpa ˈta.sa.ʔə ˈki.ɸə ˈta.mu.ka ˈa.su.hi}\endpreamble
\nogloss{\lbrack}
mapi[whole]
ʔe[\textsc{assoc}]
ai[one]
\nogloss{\rbrack}
pa[now]
tasa[chaos]
ʔe[\textsc{assoc}]
kiɸe[place]
tamuka[\textsc{cert}]
asuhi[increase]
\glft `In isolation, chaos surely follows.'\footnotemark\\
\lit{The whole is [made from] one, so chaos must increase.}\smoyd{1}{https://www.reddit.com/r/conlangs/comments/28vg3k/i_just_used_up_5_minutes_of_your_day_day_1/}
\endgl
\xe

\footnotetext{The original text was: `In an isolated system, entropy can only increase.'}

The translation of `isolated system' is not terribly transparent.
I've taken some liberties with the connotations of the phrase while translating, such that this phrase is less about literal entropy and more about the fragmentation of a group.
Countable nouns are associated with a number in the same fashion adjectives are.
Thus, \langword{mapiʔe ai} refers to a group of only one member, and therefore one which exists in isolation.

The coordinating conjunction \langword{pa} considers the second clause a direct result of the first.
This contrasts with \langword{alete,} which considers the event to have happened due to a culmination of the first among other factors.
In other words, while \langword{alete} creates a much more definitive sense of causation, \langword{pa} functions more aptly when it comes to sequencing.

Because I have reinterpreted this phrase as referring to a group, I have also described the chaos in a particular way.
Namely, considering that the members of the group are assumed to still be alive, it is the place which they meet which is considered to be chaotic.
Alternatively, the people themselves could have been described as having chaos grow within them, but this would have the connotation that the people themselves were driving the group apart rather than some external factor.

\langword{Tamuka} is used for a few reasons.
These sorts of evidentials in some contexts, can play the role of what would be rendered with modal verbs in English.
In this particular instance, \langword{tamuka} also helps cement the idea that this course of action is the only possible one (ie. all others would be therefore be completely unfathomable).

\ex
\begingl
\glpreamble menexesi meauna lu mapi muhipese
\pronounced{ˈme.nə.xə.si ˈme.au.na ˈlu ˈma.pi ˈmu.hi.pə.sə}\endpreamble
\nogloss{\lbrack}
menexesi[\textsc{pl:}star]
meauna[\textsc{pl:}moon]
lu[\textsc{obv}]
mapi[whole]
\nogloss{\rbrack}
muhipese[\textsc{pst.pfv:}tame]
\glft `The stars, the moon, they have all been blown out.'\\
\lit{The stars, the moon, they were all tamed.}\smoyd{2}{https://www.reddit.com/r/conlangs/comments/28y706/i_just_used_up_5_minutes_of_your_day_day_2/}
\endgl
\xe

The first thing to note in this translation, is that when noun phrases are not joined by \langword{ke}, and instead listed as is done here, each noun is inflected separately.
Another thing to note is that the world which \langname\ is spoken in has two moons, therefore a plural form makes more sense rather than the original singular.

The particle \langword{lu} is used so that both referents can be considered a single unit, and therefore both are considered in reference to \langword{mapi.}
Unlike other things, \langword{mapi} does not require an \langword{ʔe} to associate \langword{mapi} with any particular `whole'.
This is because there is not one particular whole being described, but rather a reference to an already existing complete group of celestial objects.

Because the stars and the moons were introduced as a patient of some verb, we may be expecting the agent to follow the verb, as it normally would in a situation with a focused patient argument.
In this particular instance however, no agent is present, and no \langword{ta} Is used to introduce any sort of reflexive reading.
\langword{Hipese} is a labile verb, meaning the sole argument is rightfully a patient (in agreement previously established case marking).
As a convenient consequence of \langword{hipese} being labile, the translated text renders much more precisely to the English.

The choice of verbs itself is notable, as this is used often in reference to exerting control over a natural forces.

\ex
\begingl
\glpreamble ɸene suɸai uʔtwaw ɸaipa ʔutawa tu akatewiʔe
\pronounced{ˈɸe.nə ˈsu.ɸai ˈuʔ.ta.wa ˈɸai.pa ˈʔu.ta.wa ˈtu ˈa.ka.te.wi.ʔə}\endpreamble
ɸene[live]
su-ɸai[without/\textsc{aug-neg}]
uʔtawa[die]
ɸaipa[unless/\textsc{neg-}pa\footnotemark]
ʔutawa[die]
tu[\textsc{2sg}]
akatewi-ʔe[warrior-ʔe]
\glft `Life is pointless... unless you die a warrior.'\\
\lit{Life will be without death, unless you die [like] a warrior.}\smoyd{3}{https://www.reddit.com/r/conlangs/comments/291tml/i_just_used_up_5_minutes_of_your_day_day_3/}
\endgl
\xe

\footnotetext{See X for a full explanation of \langword{pa.}}

\ex
\begingl
\glpreamble
\pronounced{ˈ}\endpreamble
\glft `These leviathan like creatures glimmer in the rays of the blast as they swarm and beat at the wreckage'\\
\lit{These leviathan like creatures glimmer in the rays of the blast as they swarm and beat at the wreckage.}\smoyd{4a}{https://www.reddit.com/r/conlangs/comments/294syw/i_just_used_520_minutes_of_your_day_day_4/}
\endgl
\xe

\ex
\begingl
\glpreamble
\pronounced{ˈ}\endpreamble
\glft `Outside the window is misty and dull'\\
\lit{Outside the window is misty and dull.}\smoyd{4b}{https://www.reddit.com/r/conlangs/comments/294syw/i_just_used_520_minutes_of_your_day_day_4/}
\endgl
\xe
\chapter{Passages}
\backmatter
% bibliography, glossary and index would go here.

\end{document}
